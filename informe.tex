\documentclass[10pt, a4paper]{article}
\usepackage[paper=a4paper, left=1.5cm, right=1.5cm, bottom=1.5cm, top=3.5cm]{geometry}
\usepackage[latin1]{inputenc}
\usepackage[T1]{fontenc}
\usepackage[spanish]{babel}
\usepackage{indentfirst}
\usepackage{fancyhdr}
\usepackage{latexsym}
\usepackage{lastpage}
\usepackage{xspace}
\usepackage{xargs}
\usepackage{ifthen}
\usepackage{Paquetes-Latex/texmf/tex/latex/aed2-tad/aed2-tad,Paquetes-Latex/texmf/tex/latex/aed2-symb/aed2-symb,Paquetes-Latex/texmf/tex/latex/aed2-itef/aed2-itef,Paquetes-Latex/texmf/tex/latex/aed2-tokenizer/aed2-tokenizer,Paquetes-Latex/texmf/tex/latex/caratula/caratula}
\usepackage[colorlinks=true, linkcolor=blue]{hyperref}
\usepackage{calc}
\usepackage{vmargin}
\usepackage{algorithm}
\usepackage{algpseudocode}
\usepackage{placeins}

\setpapersize{A4}
\setmargins{2.5cm}       % margen izquierdo
{1.5cm}                        % margen superior
{16.5cm}                      % anchura del texto
{23.42cm}                    % altura del texto
{10pt}                           % altura de los encabezados
{1cm}                           % espacio entre el texto y los encabezados
{0pt}                             % altura del pie de p�gina
{2cm}  

\newcommand{\f}[1]{\text{#1}}
\renewcommand{\paratodo}[2]{$\forall~#2$: #1}


% CODIGO PARA ESTILO DE MODULOS BASICOS -----------------------------------------
\newcommand{\moduloNombre}[1]{\textbf{#1}}

\let\NombreFuncion=\textsc
\let\TipoVariable=\texttt
\let\ModificadorArgumento=\textbf
\newcommand{\res}{$res$\xspace}
\newcommand{\tab}{\hspace*{7mm}}

\newcommandx{\TipoFuncion}[3]{%
  \NombreFuncion{#1}(#2) \ifx#3\empty\else $\to$ \res\,: \TipoVariable{#3}\fi%
}
\newcommand{\In}[2]{\ModificadorArgumento{in} \ensuremath{#1}\,: \TipoVariable{#2}\xspace}
\newcommand{\Out}[2]{\ModificadorArgumento{out} \ensuremath{#1}\,: \TipoVariable{#2}\xspace}
\newcommand{\Inout}[2]{\ModificadorArgumento{in/out} \ensuremath{#1}\,: \TipoVariable{#2}\xspace}
\newcommand{\Aplicar}[2]{\NombreFuncion{#1}(#2)}

\newlength{\IntFuncionLengthA}
\newlength{\IntFuncionLengthB}
\newlength{\IntFuncionLengthC}
%InterfazFuncion(nombre, argumentos, valor retorno, precondicion, postcondicion, complejidad, descripcion, aliasing)
\newcommandx{\InterfazFuncion}[9][4=true,6,7,8,9]{%
  \hangindent=\parindent
  \TipoFuncion{#1}{#2}{#3}\\%
  \textbf{Pre} $\equiv$ \{#4\}\\%
  \textbf{Post} $\equiv$ \{#5\}%
  \ifx#6\empty\else\\\textbf{Complejidad:} #6\fi%
  \ifx#7\empty\else\\\textbf{Descripci�n:} #7\fi%
  \ifx#8\empty\else\\\textbf{Aliasing:} #8\fi%
  \ifx#9\empty\else\\\textbf{Requiere:} #9\fi%
}

\newenvironment{Interfaz}{%
  \parskip=2ex%
  \noindent\textbf{\Large Interfaz}%
  \par%
}{}

\newenvironment{Representacion}{%
  \vspace*{2ex}%
  \noindent\textbf{\Large Representaci�n}%
  \vspace*{2ex}%
}{}

\newenvironment{Algoritmos}{%
  \vspace*{2ex}%
  \noindent\textbf{\Large Algoritmos}%
  \vspace*{2ex}%
}{}


\newcommand{\Titulos}[1]{
  \vspace*{1ex}\par\noindent\textbf{\large #1}\par
}

\newenvironmentx{Estructura}[2][2={estr}]{%
  \par\vspace*{2ex}%
  \TipoVariable{#1} \textbf{se representa con} \TipoVariable{#2}%
  \par\vspace*{1ex}%
}{%
  \par\vspace*{2ex}%
}%

\newboolean{EstructuraHayItems}
\newlength{\lenTupla}
\newenvironmentx{Tupla}[1][1={estr}]{%
    \settowidth{\lenTupla}{\hspace*{3mm}donde \TipoVariable{#1} es \TipoVariable{tupla}$($}%
    \addtolength{\lenTupla}{\parindent}%
    \hspace*{3mm}donde \TipoVariable{#1} es \TipoVariable{tupla}$($%
    \begin{minipage}[t]{\linewidth-\lenTupla}%
    \setboolean{EstructuraHayItems}{false}%
}{%
    $)$%
    \end{minipage}
}

\newcommandx{\tupItem}[3][1={\ }]{%
    %\hspace*{3mm}%
    \ifthenelse{\boolean{EstructuraHayItems}}{%
        ,#1%
    }{}%
    \emph{#2}: \TipoVariable{#3}%
    \setboolean{EstructuraHayItems}{true}%
}

\newcommandx{\nomoreitems}{
    \setboolean{EstructuraHayItems}{false}%
}

\newcommandx{\RepFc}[3][1={estr},2={e}]{%
  \tadOperacion{Rep}{#1}{bool}{}%
  \tadAxioma{Rep($#2$)}{#3}%
}%

\newcommandx{\Rep}[3][1={estr},2={e}]{%
  \tadOperacion{Rep}{#1}{bool}{}%
  \tadAxioma{Rep($#2$)}{true \ssi #3}%
}%

\newcommandx{\Abs}[5][1={estr},3={e}]{%
  \tadOperacion{Abs}{#1/#3}{#2}{Rep($#3$)}%
  \settominwidth{\hangindent}{Abs($#3$) \igobs #4: #2 $\mid$ }%
  \addtolength{\hangindent}{\parindent}%
  Abs($#3$) \igobs #4: #2 $\mid$ #5%
}%

\newcommandx{\AbsFc}[4][1={estr},3={e}]{%
  \tadOperacion{Abs}{#1/#3}{#2}{Rep($#3$)}%
  \tadAxioma{Abs($#3$)}{#4}%
}%


\newcommand{\DRef}{\ensuremath{\rightarrow}}

\newcommand\LONGCOMMENT[1]{%
  \hfill\#\ \begin{minipage}[t]{\eqboxwidth{COMMENT}}#1\strut\end{minipage}%
}

\newcommand{\comentario}[1]{\textcolor{cyan}{\textit{\newline $\setminus \setminus $ #1}}}

% fin CODIGO PARA ESTILO DE MODULOS BASICOS -----------------------------------------

\sloppy

\hypersetup{%
 % Para que el PDF se abra a p�gina completa.
 %pdfstartview= {FitH \hypercalcbp{\paperheight-\topmargin-1in-\headheight}},
 pdfauthor={C�tedra de Algoritmos y Estructuras de Datos II - DC - UBA},
 pdfkeywords={TADs b�sicos},
 pdfsubject={Dise�o}
}

\parskip=5pt % 10pt es el tama�o de fuente

% Pongo en 0 la distancia extra entre �temes.
\let\olditemize\itemize
\def\itemize{\olditemize\itemsep=0pt}

% Acomodo fancyhdr.
\pagestyle{fancy}
\thispagestyle{fancy}
\addtolength{\headheight}{1pt}
\cfoot{\thepage /\pageref{LastPage}}
\renewcommand{\footrulewidth}{0.4pt}

\author{}
\date{}
\title{}

% Encabezado
\lhead{Algoritmos y Estructuras de Datos II}
\rhead{Grupo 21}
% Pie de pagina
\renewcommand{\footrulewidth}{0.4pt}
\lfoot{Facultad de Ciencias Exactas y Naturales}
\rfoot{Universidad de Buenos Aires}

\begin{document}

\materia{Algoritmos y Estructuras de Datos II}
\titulo{Trabajo Pr\'actico N\'umero 2}
\subtitulo{DCNet}
\grupo{Grupo: 21}

\integrante{Alvarez, Lautaro Leonel}{268/14}{lautarolalvarez@gmail.com}
\integrante{Maddonni, Axel Ezequiel}{200/14}{axel.maddonni@gmail.com}
\integrante{Thibeault, Gabriel Eric}{114/13}{grojo94@hotmail.com}
\integrante{Vigali, Leandro Ezequiel}{951/12}{leandrovigali@yahoo.com.ar}

\maketitle

\newpage

\tableofcontents

\newpage

%MODULOS

\section{M�dulo Red}
\begin{Interfaz}

  \textbf{usa}: \tadNombre{conj($\alpha$), itConj($\alpha$), lista($\alpha$), itLista($\alpha$)}.
  
  \textbf{se explica con}: \tadNombre{Red}.

  \textbf{g\'eneros}: \TipoVariable{red}.

  \Titulos{Operaciones de Red}

  \InterfazFuncion{Computadoras}{\In{r}{red}}{conj(hostname)}
  {$res$ $=_{obs}$ dameHostnames(computadoras($r$))}%
  [O(n)]
  [devuelve el conjunto de las computadoras.]
  [res se devuelve por copia.]

  \InterfazFuncion{Conectadas?}{\In{r}{red}, \In{c1}{hostname}, \In{c2}{hostname}}{bool}
  [$c1,\ c2$ $\in$ dameHostnames(computadoras($r$))]
  {$res$ $=_{obs}$ conectadas?($r$, dameCompu($c1$), dameCompu($c2$))}
  [O(n*L)]
  [indica si las computadoras estan conectadas por alguna de sus interfaces.]
  []

  \InterfazFuncion{InterfazUsada}{\In{r}{red}, \In{c1}{hostname}, \In{c2}{hostname}}{interfaz}
  [conectadas?($r$, dameCompu($c1$), dameCompu($c2$))]
  {$res$ $=_{obs}$ interfazUsada($r$, dameCompu($c1$), dameCompu($c2$))}
  [O(n*L)]
  [devuelve la interfaz por la cual estan conectadas c1 y c2.]
  []

  \InterfazFuncion{IniciarRed}{}{red}
  [true]
  {$res$ $=_{obs}$ iniciarRed()}
  [O(1)]
  [crear una nueva Red.]
  []
  
  \InterfazFuncion{AgregarCompu}{\Inout{r}{red}, \In{c1}{compu}}{}
  [$r$ = $r_0$ $\wedge$
  (\paratodo{compu}{c}) $c$ $\in$ computadoras($r_0$) $\Rightarrow$ ip($c$) $\neq$ ip($c1$)]
  {$r$ $=_{obs}$ agregarComputadora($r_0$, $c1$)}
  [O(L+i) i=cantidad de interfaces]
  [agregar una computadora a la Red.]
  [la computadora se agrega por copia.]

  \InterfazFuncion{Conectar}{\Inout{r}{red}, \In{c_1}{hostname}, \In{i_1}{interfaz} \In{c_2}{hostname}, \In{i_2}{interfaz}}{}
  [$r$ = $r_0$ $\wedge$ $c_1, c_2$ $\in$ dameHostnames(computadoras($r$)) $\wedge$ $c_1$ $\neq$ $c_2$	$\wedge$ \newline $\neg$conectadas?($r$, dameCompu($c_1$), dameCompu($c_2$)) $\wedge$ $\neg$usaInterfaz?($r$, dameCompu($c_1$), $i_1$) $\wedge$ $\neg$usaInterfaz?($r$, dameCompu($c_2$), $i_2$) $\wedge$ $i_1$ $\in$ dameCompu($c_1$).interfaces $\wedge$ $i_2$ $\in$ dameCompu($c_2$).interfaces]
  {$r$ $=_{obs}$ (conectar($r_0$, dameCompu($c_1$), $i_1$, dameCompu($c_2$), $i_2$)}
  [O(n$^{6}$ + n$^{5}$ * L)]
  [conectar dos computadoras de la red.]
  []
   
  \InterfazFuncion{Vecinos}{\In{r}{red}, \In{c}{hostname}}{conj(hostname)}
  [$c$ $\in$ dameHostnames(computadoras($r$))]
  {$res$ $=_{obs}$ dameHostnames(vecinos($r$, dameCompu($c$)))}
  [O(n*L)]
  [da el conjunto de computadoras vecinas.]
  [el conjunto se devuelve por copia.]

  \InterfazFuncion{UsaInterfaz?}{\In{r}{red}, \In{c}{hostname}, \In{i}{interfaz}}{bool}
  [$c$ $\in$ dameHostnames(computadoras($r$))]
  {$res$ $=_{obs}$ usaInterfaz?($r$, dameCompu($c$), $i$)}
  [O(n)]
  [indica si la interfaz est\'a siendo utilizada.]
  []

  \InterfazFuncion{CaminosMinimos}{\In{r}{red}, \In{c_1}{hostname}, \In{c_2}{hostname}}{conj(lista(hostname))}
  [$c1,\ c2$ $\in$ dameHostnames(computadoras($r$))]
  {$res$ $=_{obs}$ dameCaminosdeHostnames(caminosMinimos($r$, dameCompu($c_1$), dameCompu($c_2$)))}
  [O(n*L)]
  [devuelve los conjuntos de caminos minimos entre las computadoras ingresadas.]
  [res se devuelve por copia.]
  
  \InterfazFuncion{HayCamino?}{\In{r}{red}, \In{c_1}{hostname}, \In{c_2}{hostname}}{bool}
  [$c1,\ c2$ $\in$ dameHostnames(computadoras($r$))]
  {$res$ $=_{obs}$ hayCamino?($r$, dameCompu($c_1$), dameCompu($c_2$))}
  [O(n*n)]
  [indica si las computadoras son alcanzables mediante alg\'un camino.]
  []
  
  \InterfazFuncion{$\bullet == \bullet$}{\In{r_1}{red}, \In{r_2}{red}}{bool}
  [true]
  {$res$ =$_{obs}$ ($r_1$ =$_{obs}$ $r_2$}%
  [O(n*n * ( L + n*n + m ) + n*m*m)]
  [indica si dos redes son iguales.]
  [] 
  
  \InterfazFuncion{Copiar}{\In{r}{red}}{red}
  {$res$ =$_{obs}$ r}
  []
  [O(n$^{3}$)]
  [copia la red.]
  [res se devuelve por copia]
  
   \textbf{donde:} \newline 
   \TipoVariable{hostname} es \TipoVariable{string}, \newline
   \TipoVariable{interfaz} es \TipoVariable{nat}, \newline
   \TipoVariable{compu} es \TipoVariable{tupla}< ip: \TipoVariable{hostname}, interfaces: \TipoVariable{conj}(\TipoVariable{interfaz})>.
   
\end{Interfaz}

\textbf{} %dejo un espacio

\textbf{Especificaci\'on de las operaciones auxiliares utilizadas en la interfaz (no exportadas)}

\begin{tad}{\tadNombre{Red extendida}}

\tadExtiende{\tadNombre{Red}}

\setlength{\tadAnchoTipoFunciones}{0pt}

\tadOtrasOperaciones 

\tadOperacion{damehostnames}{conj(compu)}{conj(hostname)}{}
\tadOperacion{dameCompu}{red /r, hostname /s}{compu}{$s$ $\in$ hostnames($r$)}
\tadOperacion{auxDameCompu}{red /r, hostname /s, conj(compu) /cc}{compu}{$s$ $\in$ hostnames($r$) $\wedge$ $cc$ $\subset$ computadoras($r$)}
\tadOperacion{dameCaminosDeHostnames}{conj(secu(compu))}{conj(secu(hostname))}{}
\tadOperacion{dameSecuDeHostnames}{secu(compu)}{secu(hostname)}{}

\tadAxiomas[\paratodo{red}{r}, \paratodo{conj(compu)}{cc}, \paratodo{hostname}{s}, \paratodo{conj(secu(compu))}{cs}, \paratodo{secu(compu)}{secu}]

\tadAxioma{dameHostnames($cc$)}
{ \IF {vacio?($cc$)} 
THEN {$\emptyset$}
ELSE {Ag( ip(dameUno($cc$)), dameHostnames(sinUno($cc$)) )}
FI}

\tadAxioma{dameCompu($r$, $s$)}{auxDameCompu($r$, $s$, computadoras($r$))}

\tadAxioma{auxDameCompu($r$, $s$, $cc$)}
{ \IF {ip(dameUno($cc$)) = $s$} 
THEN {dameUno($cc$)}
ELSE {auxDameCompu($r$, $s$, sinUno($cc$))}
FI}

\tadAxioma{dameCaminosDeHostnames($cs$)}
{ \IF {vacio?($cs$)} 
THEN {$\emptyset$}
ELSE {Ag( dameSecuDeHostnames(dameUno($cs$)), dameCaminosDeHostnames(sinUno($cs$)) )}
FI}

\tadAxioma{dameSecuDeHostnames($secu$)}
{ \IF {vacia?($secu$)} 
THEN {$<>$}
ELSE {ip(prim($secu$)) $\bullet$ dameSecuDeHostnames(fin($secu$)) }
FI}

\end{tad}

\begin{Representacion}

\begin{Estructura}{red}[estr\_red]

\textbf{donde:} \newline 
\TipoVariable{estr\_red} es \TipoVariable{dicc(hostname, datos)} \newline
   
\begin{Tupla}[datos]
\tupItem{interfaces}{\TipoVariable{conj(interfaz)}} \newline \nomoreitems{}	 
\tupItem{conexiones}{\TipoVariable{dicc(interfaz, hostname)}} \newline \nomoreitems{}
\tupItem{alcanzables}{\TipoVariable{dicc(dest: hostname, caminos: conj(lista(hostname)))}}
\end{Tupla}

\TipoVariable{hostname} es \TipoVariable{string}, \TipoVariable{interfaz} es \TipoVariable{nat}.

\end{Estructura}
\Comment{Al no tener que cumplir con complejidades utilizamos un diccionario con los hostnames como claves. El significado de cada hostname corresponde a una tupla con datos de la computadora con ese hostname.}\newline

\comentario{Rep en Castellano: \newline
Para cada computadora: \newline
1: Las interfaces usadas pertenecen al conjunto de interfaces de la compu.\newline
2: Los vecinos pertenecen a las computadoras de la red.\newline
3: Los vecinos son distintos a la compu actual.\newline
4: Los vecinos no se repiten.\newline
5: Las conexiones son bidireccionales.\newline
6: Los alcanzables pertenecen a las computadoras de la red.\newline
7: Los alcanzables son distintos a la actual.\newline
8: Los alcanzables tienen un camino v\'alido hacia ellos desde la actual.\newline
9: Para cada alcanzable, el conjunto de camiinos v\'alidos no es vac\'io.\newline
10: Todos los caminos en el diccionario alcanzables son v\'alidos.\newline
11: Los caminos son m\'inimos.\newline
12: Est\'an todos los m\'inimos.
}

\tadAlinearFunciones{REP}{estr\_red /e}

\Rep[estr\_red]{
	($\forall$ $c$: hostname, $c$ $\in$ claves($e$)) \textcolor{red}{(} 
	\comentario{1} claves(obtener($e$, $c$).conexiones) $\subseteq$ obtener($e$, $c$).interfaces $\wedge$ \newline
	 ($\forall$ $i$: interfaz, $i$ $\in$ claves(obtener($e$, $c$).conexiones)) \textcolor{blue}{(}  
	\comentario{2} obtener(obtener($e$, $c$).conexiones, $i$) $\in$ claves($e$) $\wedge$ 	
	\comentario{3} obtener(obtener($e$, $c$).conexiones, $i$) $\neq$ $c$ 	$\wedge$ 
	\comentario{4} ($\neg$ $\exists$ $i\sp{\prime}$: interfaz, $i\sp{\prime}$ $\in$ claves(obtener($e$, $c$).conexiones), $i$ $\neq$ $i\sp{\prime}$) obtener(obtener($e$, $c$).conexiones, $i$) == obtener(obtener($e$, $c$).conexiones, $i\sp{\prime}$) $\wedge$ 
	\comentario{5} ($\forall$ $h$: hostname) ( $h$ == obtener(obtener($e$, $c$).conexiones, $i$)
	$\Rightarrow$ ($\exists$ $i\sp{\prime}$:int) obtener(obtener($e$, $h$).conexiones, $i\sp{\prime}$) == $c$ ) 
	\textcolor{blue}{)} $\wedge$ 
	\comentario{6} claves(obtener($e$, $c$).alcanzables $\subseteq$ claves($e$) $\wedge$ \newline
	 ($\forall$ $a$: hostname, $a$ $\in$ claves(obtener($e$, $c$).alcanzables) \textcolor{blue}{(} 
	\comentario{7} $a$ $\neq$ $c$ $\wedge$ 
	\comentario{8} ($\exists$ $s$: secu(hostname)) esCaminoV\'alido($c$, $a$, $s$)  $\wedge$ 
	\comentario{9} $\#$obtener(obtener($e$, $c$).alcanzables, $a$) > 0 $\wedge_L$ \newline
	($\forall$ $camino$: secu(hostname), $camino$ $\in$ obtener(obtener($e$, $c$).alcanzables, $a$) (
	\comentario{10} esCaminoV\'alido($c$, $a$, $camino$) $\wedge$ 
	\comentario{11} $\neg$($\exists$ $camino\sp{\prime}$: secu(hostname), $camino$ $\neq$ $camino\sp{\prime}$, esCaminoV\'alido($c$, $a$, $camino\sp{\prime}$)) long($camino\sp{\prime}$) < long($camino$ ) $\wedge$ 
	\comentario{12} $\neg$($\exists$ $camino\sp{\prime}$: secu(hostname), $camino$ $\neq$ $camino\sp{\prime}$, esCaminoV\'alido($c$, $a$, $camino\sp{\prime}$), long($camino$ == long($camino\sp{\prime}$)) ($camino\sp{\prime}$ $\notin$ obtener(obtener($e$, $c$).alcanzables, $a$) )
	\newline \textcolor{blue}{)} \textcolor{red}{)}
}

La abreviatura $esCaminoValido$ usada en el Rep se debe leer: (no son funciones, son abreviaturas para hacer m\'as f\'acil la lectura)

\tadAxioma{$esCaminoValido$ ($orig$, $dest$, $secu$)} {  
	( prim($secu$) == $orig$ $\wedge$ \newline
	($\forall$ $i$: nat, 0 $<$ $i$ $<$ long($secu$) ) esVecino ($secu[i]$, $secu[i+1]$) $\wedge$ \newline
	$secu$ [long($secu$)-1] == $dest$ $\wedge$ \newline
	sinRepetidos($secu$) )
	}

Con $esVecino$ ($h1$, $h2$) $\equiv$  ($\exists$ $i$: interfaz) $h2$ == obtener (obtener($e$, $h1$).conexiones, $i$) 

\textbf{}
\textbf{}

\AbsFc[estr\_red]{red}[e]{$r$ $|$ computadoras ($r$) = dameComputadoras($e$) $\wedge_L$ \newline
	($\forall$ $c1$, $c2$: compu, $c1$, $c2$ $\in$ computadoras($r$)) conectadas?($r$, $c1$, $c2$) = ($\exists$ $i$: interfaz) ( $c2$.ip = obtener(obtener($e$, $c1$.ip).conexiones, $i$) ) $\wedge_L$ \newline
	interfazUsada($r$, $c1$, $c2$) = buscarClave (obtener($e$, $c1$.ip).conexiones, $c2$.ip)	
	}

\textbf{} %dejo un espacio

\newpage

\textbf{Especificaci\'on de las funciones auxiliares utilizadas en abs}

\tadAlinearFunciones{auxDameComputadoras}{dicc(interfaz, hostname), hostname, conj(interfaz)}

\tadOperacion{dameComputadoras}{dicc(hostname; X)}{conj(computadoras)}{}
\tadOperacion{auxDameComputadoras}{dicc(hostname; X), conj(hostname)}{conj(computadoras)}{}
\tadOperacion{buscarClave}{dicc(interfaz; hostname), hostname}{interfaz}{}
\tadOperacion{auxBuscarClave}{dicc(interfaz; hostname), hostname, conj(interfaz)}{interfaz}{}

\tadAxiomas[\paratodo{dicc(hostname, X)}{e}, \paratodo{dicc(interfaz, hostname)}{d}, \paratodo{conj(hostname)}{cc}, \paratodo{conj(interfaz)}{ci}, \paratodo{hostname}{h}]

\tadAxioma{dameComputadoras($e$)}{auxDameComputadoras($e$, claves($e$))}
\tadAxioma{auxDameComputadoras($e$, $cc$)}
{ \IF {$\emptyset$?($cc$)} 
THEN {$\emptyset$}
ELSE {Ag( <dameUno($cc$), obtener($e$, dameUno($cc$)).interfaces>, auxDameComputadoras($e$, sinUno($cc$)) )}
FI}

\tadAxioma{buscarClave($d$, $h$)}{auxBuscarClave($d$, $h$, claves($d$))}
\tadAxioma{auxBuscarClave($d$, $h$, $ci$)}
{ \IF {obtener($d$, dameUno($cc$)) = $h$} 
THEN {dameUno($cc$)}
ELSE {auxBuscarClave($d$, $h$, sinUno($ci$))}
FI}

\end{Representacion}

\newpage

\begin{Algoritmos}
\begin{algorithm}
\caption{Implementaci\'on de Computadoras}
\begin{algorithmic}[0]
\Function{iComputadoras}{in r: estr\_red}{$\rightarrow$ res: conj(hostname)}
	\State it $\gets$ crearIt(r) \Comment{O(1)}
	\State res $\gets$ Vacio() \Comment{conjunto} \Comment{O(1)}
	\While{HaySiguiente(it)} \Comment{Guarda: O(1)} \Comment{El ciclo se ejecuta n veces} \Comment{O(n)}
		\State Agregar(res, SiguienteClave(it)) \Comment{O(1)}
		\State Avanzar(it) \Comment{O(1)}
	\EndWhile
\EndFunction \Comment{\textbf{O(n)}}
\end{algorithmic}
\end{algorithm}

\begin{algorithm}
\caption{Implementaci\'on de Conectadas?}
\begin{algorithmic}[0]
\Function{iConectadas?}{in r: estr\_red, in c1: hostname, in c2: hostname}{$\rightarrow$ res: bool}
	\State it $\gets$ CrearIt(Significado(r,c1).conexiones) \Comment{O(n)}
	\State res $\gets$ FALSE \Comment{O(1)}
	\While{HaySiguiente(it) $\&\&$ $\neg$res} \Comment{Guarda: O(1)} \Comment{El ciclo se ejecuta a lo sumo n-1 veces} \Comment{O(n)}
		\If{SiguienteClave(it)==c2} \Comment{O(L)}
			\State res $\gets$ TRUE
		\EndIf
		\State Avanzar(it) \Comment{O(1)}
	\EndWhile
\EndFunction \Comment{\textbf{O(n*L)}}
\end{algorithmic}
\end{algorithm}

\begin{algorithm}
\caption{Implementaci\'on de InterfazUsada}
\begin{algorithmic}[0]
\Function{iInterfazUsada}{in r: estr\_red, in c1: hostname, in c2: hostname}{$\rightarrow$ res: interfaz}
	\State it $\gets$ CrearIt(significado(r,c1).conexiones) \Comment{O(n)}
	\While{HaySiguiente(it)} \Comment{Guarda: O(1)} \Comment{El ciclo se ejecuta a lo sumo n-1 veces} \Comment{O(n)}
		\If{SiguienteSignificado(it)==c2} \Comment{O(L)}
			\State res $\gets$ SiguienteClave(it) \Comment{nat por copia} \Comment{O(copiar(nat))}
		\EndIf
		\State Avanzar(it) \Comment{O(1)}
	\EndWhile
\EndFunction \Comment{\textbf{O(n*L)}}
\end{algorithmic}
\end{algorithm}

\begin{algorithm}
\caption{Implementaci\'on de IniciarRed}
\begin{algorithmic}[0]
\Function{iIniciarRed()}{}{$\rightarrow$ res: estr\_red}
	\State res $\gets$ Vacio() \Comment{Diccionario} \Comment{O(1)}
\EndFunction \Comment{\textbf{O(1)}}
\end{algorithmic}
\end{algorithm}

\begin{algorithm}
\caption{Implementaci\'on de AgregarCompu}
\begin{algorithmic}[0]
\Function{iAgregarCompu}{inout r: estr\_red, in c1: compu}{}
	\State nuevoDiccVacio $\gets$ Vacio() \Comment{Diccionario} \Comment{O(1)}
	\State DefinirRapido(r, c1.ip, tupla(Copiar(c1.interfaces), nuevoDiccVacio, nuevoDiccVacio)) \Comment{O(Copiar(sL) + Copiar(conj(interfaz)) con sL=string de largo L}
\EndFunction \Comment{\textbf{O(L + i) con i=cantidad de interfaces}}
\end{algorithmic}
\end{algorithm}

\begin{algorithm}
\caption{Implementaci\'on de Conectar}
\begin{algorithmic}[0]
\Function{iConectar}{inout r: estr\_red, in c1: hostname, int i1:interfaz, in c2: hostname, in i2:interfaz}{}
\comentario{Actualizo conexiones de ambas}
\State DefinirRapido(Significado(r, c1).conexiones, i1, c2) \Comment{O(n) + O(L) + O(copiar(nat))}
\State DefinirRapido(Significado(r, c2).conexiones, i2, c1) \Comment{O(n) + O(L) + O(copiar(nat))} 
\comentario{Actualizo caminos de ambas}
\State ActualizarCaminos(r, c1, c2) \Comment{O(n$^{4}$ + n$^{3}$ x L)}
\State ActualizarCaminos(r, c2, c1) \Comment{O(n$^{4}$ + n$^{3}$ x L)}
\comentario{Creo conjunto con los actualizados hasta el momento}
\State actualizados $\gets$ Vacio() \Comment{conjunto}
\State AgregarRapido(actualizados, c1) \Comment{O(L)}
\State AgregarRapido(actualizados, c2) \Comment{O(L)}
\comentario{Actualizo caminos del resto de la Red por recursion}
\State ActualizarVecinos(r, c1, actualizados)  \Comment{O(n$^{6}$ + n$^{5}$ * L)}
\EndFunction \Comment{O(n$^{6}$ + n$^{5}$ * L)}
\end{algorithmic}
\end{algorithm}

\begin{algorithm}
\caption{Implementaci\'on de funci\'on auxiliar Actualizar Caminos}
\begin{algorithmic}[0]
\Function{iActualizarCaminos}{inout r: estr\_red, in c1: hostname, in c2: hostname}{}
\comentario{Actualiza los caminos de c1 con los de c2}
\comentario{Recorro los alcanzables de c2}
\State itAlcanzables2 $\gets$ crearIt(Significado(r, c2).alcanzables) \Comment{O(n)}
\While {(HaySiguiente(itAlcanzables2)} \Comment{se ejecuta a lo sumo n-1 veces} \Comment{O(n) x}
	\comentario{Recorro alcanzables de c1}
	\State itAlcanzables1 $\gets$ crearIt(Significado(r, c1).alcanzables) \Comment{O(n)}
	\While{(HaySiguiente(itAlcanzables1))} \Comment{se ejecuta a lo sumo n-1 veces} \Comment{O(n) x}
		\If{(SiguienteClave(itAlcanzables2) == SiguienteClave(itAlcanzables1))} \Comment{O(L)}
			\comentario{El alcanzable ya estaba, me fijo que caminos son m\'as cortos}
			\State itCaminos $\gets$ crearIt (SiguienteSignificado(itAlcanzables2)) \Comment{O(1)}
			\State camino2 $\gets$ Siguiente(itCaminos) \Comment{camino minimo del c2} \Comment{O(n)}
			\State itCaminos $\gets$ crearIt (SiguienteSignificado(itAlcanzables1)) \Comment{O(1)}
			\State camino1 $\gets$ Siguiente(itCaminos) \Comment{camino minimo del c1} \Comment{O(n)}
			
			\If{(longitud(camino1) > longitud(camino2))} \Comment{cada camino tiene a lo sumo n elementos} \Comment{O(1)}
				\comentario{Los caminos nuevos son mas cortos, borro los que están y copio los nuevos}
				\State Borrar(Significado(r, c1).alcanzables, SiguienteClave(itAlcanzables1)) \Comment{O(n)}
				\comentario{Nuevo alcanzable: me copio los caminos agregando c1 al principio}
				\State itCaminos $\gets$ crearIt (SiguienteSignificado(itAlcanzables2)) \Comment{O(1)}
				\State caminos $\gets$ Vacio() \Comment{conjunto donde voy a guardar los caminos modificados} \Comment{O(1)}
				\While { (HaySiguiente(itCaminos))} \Comment{O(n) x}
					\State nuevoCamino $\gets$ copy(Siguiente(itCaminos)) \Comment{copio el camino que voy a modificar} \Comment{O(n)}
					\State AgregarAdelante(nuevoCamino, c1) \Comment{O(L)}
					\State AgregarRapido (caminos, nuevoCamino) \Comment{O(n)}
					\State Avanzar(itCaminos) \Comment{O(1)}
				\EndWhile  \Comment{O(n x (n + L))}
				\comentario {agrego el nuevo alcanzable con el camino}
				\State DefinirRapido(Significado(r,c1).alcanzables, SiguienteClave(itAlcanzables2), caminos) \Comment{O(n) + O(L)}
			
			\Else 
				\If {(longitud(camino1) == longitud(camino2))} \Comment{O(1)}
					\comentario{ Tengo que agregar los nuevos caminos (modificados) al conjunto de caminos actual}
					\State itCaminos $\gets$ crearIt(SiguienteSignificado(itAlcanzables2)) \Comment{O(1)}
					\While {(HaySiguiente(itCaminos))} \Comment{O(n) x}
						\State nuevoCamino $\gets$ copy(Siguiente(itCaminos)) \Comment {copio el camino que voy a modificar} \Comment{O(n)}
						\State AgregarAdelante(nuevoCamino, c1) \Comment{O(L)}
						\State Agregar(SiguienteSignificado(itAlcanzables1), nuevoCamino) \Comment{O(n)}
						\State Avanzar(itCaminos) \Comment{O(1)}
					\EndWhile  \Comment{O(n x (n + L))}
				\EndIf
			\EndIf
		\Else
			\comentario{Nuevo alcanzable : me copio los caminos agregando c1 al principio}
			\State itCaminos $\gets$ crearIt (SiguienteSignificado(itAlcanzables2)) \Comment{O(1)}
			\State caminos $\gets$ Vacio() \Comment{conjunto donde voy a guardar los caminos modificados} \Comment{O(1)}
			\While { (HaySiguiente(itCaminos))}\Comment{O(n) x}
				\State nuevoCamino $\gets$ copy(Siguiente(itCaminos)) \Comment{copio el camino que voy a modificar} \Comment{O(n)}
				\State AgregarAdelante(nuevoCamino, c1) \Comment{O(L)}
				\State AgregarRapido (caminos, nuevoCamino)\Comment{O(n)}
				\State Avanzar(itCaminos) \Comment{O(1)}
			\EndWhile  \Comment{O(n x (n + L))}
%			\comentario {agrego el nuevo alcanzable con el camino}
			\State DefinirRapido(Significado(r, c1).alcanzables, SiguienteClave(itAlcanzables2), caminos)
		\EndIf
		
			
	\algstore {actualizarcaminos}						

\end{algorithmic}
\end{algorithm}


\begin{algorithm}
\begin{algorithmic}[0]

	\algrestore {actualizarcaminos}
	
		\State Avanzar(itAlcanzables1) \Comment{O(1)}
	\EndWhile  \Comment{O(n x (n + (n x (n + L))) )}
	\State Avanzar(itAlcanzables2)\Comment{O(1)}
\EndWhile {O(n + n x (n x (n + (n x (n + L)))}
\EndFunction \Comment{O(n$^{4}$ + n$^{3}$ x L)}
\end{algorithmic}
\end{algorithm}		
				
				
\begin{algorithm}
\caption{Implementaci\'on de funci\'on auxiliar Actualizar Vecinos}
\begin{algorithmic}[0]
\Function{iActualizarVecinos}{inout r: estr\_red, in c1: hostname, in actualizados: conj(hostname) }{}
\comentario{Actualiza los caminos de los vecinos de C, y luego hace recursion para los vecinos de los vecinos.}
\State itVecinos $\gets$ crearIt(Significado(r, c1).conexiones) \Comment{O(n)}
\While{ ( HaySiguiente(itVecinos) )} \Comment{O(n x)}
	\comentario{ Si todavía no fue actualizado, lo actualizo y hago recursi\'on sobre los vecinos.}
	\If{(SiguienteClave(itVecinos) $\notin$ actualizados)} \Comment{O(n x L)}
		\State ActualizarCaminos(r, SiguienteClave(itVecinos), c )  \Comment{O(n$^{4}$ + n$^{3}$ x L)}
		\State AgregarRapido (actualizados, SiguienteClave(itVecinos)) \Comment{O(L)}
		\State ActualizarVecinos(r, SiguienteClave(itVecinos), actualizados) 
		\Comment{recursi\'on hasta actualizar las n computadoras, O(n) x}
	\EndIf
\State Avanzar(itVecinos) \Comment{O(1)}
\EndWhile
\EndFunction \Comment{O(n$^{6}$ + n$^{5}$ * L)}
\end{algorithmic}
\end{algorithm}

\begin{algorithm}
\caption{Implementaci\'on de Vecinos}
\begin{algorithmic}[0]
\Function{iVecinos}{inout r: estr\_red, in c1: hostname}{$\rightarrow$ res: conj(hostname)}
	\State it $\gets$ CrearIt(Significado(r,c1).conexiones) \Comment{O(1) + O(n)}
	\State res $\gets$ Vacio() \Comment{Conjunto} \Comment{O(1)}
	\While{HaySiguiente(it)} \Comment{Guarda: O(1)} \Comment{El ciclo se ejecuta a lo sumo n-1 veces} \Comment{O(n)}
		\State AgregarRapido(res, SiguienteSignificado(it)) \Comment{O(L)}
		\State Avanzar(it) \Comment{O(1)}
	\EndWhile
\EndFunction \Comment{\textbf{O(n*L)}}
\end{algorithmic}
\end{algorithm}

\begin{algorithm}
\caption{Implementaci\'on de UsaInterfaz}
\begin{algorithmic}[0]
\Function{iUsaInterfaz}{in r: estr\_red, in c: hostname, in i: interfaz}{$\rightarrow$ res: bool}
	\State res $\gets$ Definido?(Significado(r,c).conexiones,i) \Comment{O(comparar(nat)*n)}
\EndFunction \Comment{\textbf{O(n)}}
\end{algorithmic}
\end{algorithm}

\begin{algorithm}
\caption{Implementaci\'on de CaminosMinimos}
\begin{algorithmic}[0]
\Function{iCaminosMinimos}{in r: estr\_red, in c1: hostname, in c2: hostname}{$\rightarrow$ res: conj(lista(hostname))}
	\State itCaminos $\gets$ crearIt(Significado(Significado(r,c1).alcanzables, c2)) \Comment{O(1) + O(L*n)}
	\State res $\gets$ Vacio() \Comment{Conjunto} \Comment{O(1)}
	\While{HaySiguiente(itCaminos)}
		\State AgregarRapido(res, Siguiente(itCaminos)) \Comment{O(1)}
		\State Avanzar(itCaminos) \Comment{O(1)}	
	\EndWhile
\EndFunction \Comment{\textbf{O(n*L)}}
\end{algorithmic}
\end{algorithm}

\begin{algorithm}
\caption{Implementaci\'on de HayCamino?}
\begin{algorithmic}[0]
\Function{iHayCamino?}{in r: estr\_red, in c1: hostname, in c2: hostname}{$\rightarrow$ res: bool}
	\State res $\gets$ Definido?(Significado(r,c1).alcanzables, c2) \Comment{O(n*n)}
\EndFunction \Comment{\textbf{O(n*n)}}
\end{algorithmic}
\end{algorithm}

\begin{algorithm}
\caption{Implementaci\'on de ==}
\begin{algorithmic}[0]
\Function{iIgualdad}{in r1: estr\_red, in r2: estr\_red}{$\rightarrow$ res: bool}
	\State res $\gets$ TRUE	\Comment{O(1)}
	\If{$\neg$($\#$Claves(r1)==$\#$Claves(r2))} \Comment{O(comparar(nat))}
\comentario{Si la cantidad de claves son distintas => las redes son distintas}
		\State res $\gets$ FALSE \Comment{O(1)}
	\Else
		\State itRed1 $\gets$ CrearIt(r1) \Comment{O(1)}
		\While{HaySiguiente(itRed1) $\&\&$ res} \Comment{Guarda: O(1)} \Comment{Se ejecuta n veces} \Comment{O(n)}
\comentario{Recorro la red 1 y me fijo para cada una de sus computadoras}
			\If{$\neg$(Definido?(r2, SiguienteClave(itRed1))} \Comment{O(L*n)}
\comentario{Si no está definido su hostname en la red 2 => las redes son distintas}
				\State res $\gets$ FALSE \Comment{O(1)}
			\Else
				\State Compu2 $\gets$ Significado(r2,SiguienteClave(itRed1)) \Comment{O(L*n)}
				\State Compu1 $\gets$ SiguienteSignificado(itRed1) \Comment{O(1)}
\comentario{Tomo las computadoras de red 1 y red 2 con el mismo hostname y las comparo}
				\If{$\neg$(Comp1.interfaces == Comp2.interfaces)} \Comment{O(m*m) con m=cantidad de interfaces}
\comentario{Si sus interfaces son distintas => las redes son distintas}
					\State res $\gets$ FALSE \Comment{O(1)}
				\EndIf
				
				\If{$\neg$(Comp1.conexiones == Comp2.conexiones)}
\comentario{Si sus conexiones son distintas => las redes son distintas}
					\State res $\gets$ FALSE \Comment{O(1)}
				\EndIf
				\If{$\neg$($\#$Claves(Compu1.alcanzables)==$\#$Claves(Compu2.alcanzables))}
\comentario{Si sus cantidades de alcanzables son distintas => las redes son distintas}
					\State res $\gets$ FALSE \Comment{O(1)}
				\Else
					\State itAlc1 $\gets$ CrearIt(Compu1.alcanzables) \Comment{O(1)}
					\While{HaySiguiente(itAlc1) $\&\&$ res} \Comment{se ejecuta a lo sumo n$-$1 veces} \Comment{O(n)}
\comentario{Para cada alcanzable de la computadora de la red 1}
						\If{$\neg$(Definido?(Comp2.alcanzables, SiguienteClave(itAlc1)))} \Comment{O(m)}
\comentario{Si no est\'a definida en los alcanzables de la compu de la red 2 => las redes son distintas}
							\State res $\gets$ FALSE
						\Else
							\State Caminos1 $\gets$ SiguienteSignificado(itAlc1) \Comment{O(1)}
							\State Caminos2 $\gets$ Significado(Comp2.alcanzables, itAlc1) \Comment{O(n)}
\comentario{Me guardo los 2 conjuntos de caminos (de la compu de la red 1 y la de la red 2)}
							
	\algstore {igualdadRed}						

\end{algorithmic}
\end{algorithm}


\begin{algorithm}
\begin{algorithmic}[0]

\algrestore {igualdadRed}					
					\If{$\neg$(Longitud(Caminos1) == Longitud(Caminos2))} \Comment{O(comparar(nat))}
\comentario{Si sus cantidades son distintas => las redes son distintas}
								\State res $\gets$ FALSE
							\Else
								\State itCaminos1 $\gets$ CrearIt(Caminos1) \Comment{O(1)}
								\While{HaySiguiente(itCaminos1) $\&\&$ res}
\comentario{Para cada camino en el conjunto de caminos de la compu de la red 1}
\comentario{Recorro los caminos de la compu de la red 2}
									\State itCaminos2 $\gets$ CrearIt(Caminos2) \Comment{O(1)}
									\State noEncontro $\gets$ TRUE \Comment{O(1)}
									\While{HaySiguiente(itCaminos2) $\&\&$ noEncontro}
\comentario{Busco que el camino de la compu de la red 1 est\'e en la compu de la red 2}
										\If{Siguiente(itCaminos2) == Siguiente(ItCaminos1)}
											\State noEncontro $\gets$ FALSE
										\EndIf
										\State Avanzar(itCaminos2) \Comment{O(1)}
									\EndWhile
									\If{noEncontro} \Comment{O(1)}
\comentario{Si no encontr\'o alguno => las redes son distintas}
										\State res $\gets$ FALSE \Comment{O(1)}
									\EndIf
									\State Avanzar(itCaminos1) \Comment{O(1)}
								\EndWhile
							\EndIf
						\EndIf
						\State Avanzar(itAlc1) \Comment{O(1)}
					\EndWhile
				\EndIf
			\EndIf
			\State Avanzar(itRed1) \Comment{O(1)}
		\EndWhile
	\EndIf
\EndFunction \Comment{\textbf{O(n*n * ( L + n*n + m ) + n*m*m)}}
\end{algorithmic}
\end{algorithm}

\begin{algorithm}
\caption{Implementaci\'on de Copiar}
\begin{algorithmic}[0]
\Function{iCopiar}{in r: estr\_red}{$\rightarrow$ res: red}
	\State res $\gets$ IniciarRed() \Comment{O(1)}
\comentario{Crea una red vacia.}
	\State itRed $\gets$ CrearIt(r) \Comment{O(1)}
	\While{HaySiguiente(itRed)} \Comment{O(1)} \Comment{se ejecuta n veces} \Comment{O(n)}
\comentario{Para cada computadora en la red original.}
		\State copiaAlcanzables $\gets$ Vacio() \Comment{diccionario} \Comment{O(1)}
\comentario{Inicia los alcanzables en vacio.}
		\State itAlcanzables $\gets$ CrearIt(SiguienteSignificado(itRed).alcanzables) \Comment{O(1)}
		\While{HaySiguiente(itAlcanzables)} \Comment{Guarda: O(1)} \Comment{se ejecuta a lo sumo n veces} \Comment{O(n)}
\comentario{Para cada conjunto de caminos m\'inimos (cada destino).}
			\State copiaCaminos $\gets$ Vacia() \Comment{lista} \Comment{O(1)}
\comentario{Inicia el conjunto de caminos m\'inimos como vac\'io.}
			\State itCaminos $\gets$ CrearIt(SiguienteSignificado(itAlcanzables)) \Comment{O(1)}
			\While{HaySiguiente(itCaminos)}
\comentario{Para cada camino en el conjunto original.}
				\State AgregarAdelante(copiaCaminos, Siguiente(itCaminos)) \Comment{O(copiar(camino))}
\comentario{Copia el camino original y lo agrega adelante del conjunto de caminos m\'inimos.}
				\State Avanzar(itCaminos) \Comment{O(1)}
			\EndWhile			
			\State Definir(copiaAlcanzables, SiguienteClave(itAlcanzables), copiaCaminos)
\comentario{Define el destino y sus caminos m\'inimos en la copia de alcanzables.}
			\State Avanzar(itAlcanzables) \Comment{O(1)}
		\EndWhile
		\State Definir(res, SiguienteClave(itRed), Tupla(Copiar(SiguienteSignificado(itRed).interfaces), Copiar(SiguienteSignificado(itRed).conexiones), copiaAlcanzables))
\comentario{Define la copia de la computadora con los campos antes copiados.}
		\State Avanzar(itRed) \Comment{O(1)}
	\EndWhile
\EndFunction
\end{algorithmic}
\end{algorithm}


\end{Algoritmos}

\newpage




\clearpage

\section{M�dulo DCNet}
\begin{Interfaz}

  \textbf{usa}: \tadNombre{red, conj($\alpha$), itConj($\alpha$), lista($\alpha$), itLista($\alpha$), dicc$_{UNIV}$($\kappa$, $\sigma$), dicc$_{LOG}$($\kappa$, $\sigma$), cola$_{LOG}$($\alpha$)}.
  
  \textbf{se explica con}: \tadNombre{DCNet}.

  \textbf{g\'eneros}: \TipoVariable{dcnet}.

  \Titulos{Operaciones de DCNet}

  \InterfazFuncion{Red}{\In{d}{dcnet}}{red}
  {alias($res$ =$_{obs}$ red($d$))}%
  [O(1)]
  [devuelve la red asociada.]
  [res no es modificable.]

  \InterfazFuncion{CaminoRecorrido}{\In{d}{dcnet}, \In{p}{IDpaquete}}{lista(hostname)}
  [IDpaqueteEnTransito?($d$, $p$)]
  {$res$ =$_{obs}$ dameSecuDeHostnames(caminoRecorrido($d$, damePaquete($p$)))}%
  [O(n * log(k))]
  [devuelve el camino recorrido desde el origen hasta el actual.]
  [res se devuelve por copia.]
  
  \InterfazFuncion{CantidadEnviados}{\In{d}{dcnet}, \In{c}{hostname}}{nat}
  [$c$ $\in$ dameHostnames(computadoras(red($d$)))]
  {$res$ =$_{obs}$ cantidadEnviados($d$, dameCompu($c$))}%
  [O(L)]
  [devuelve la cantidad de paquetes enviados por la computadora.]
  []

  \InterfazFuncion{EnEspera}{\In{d}{dcnet}, \In{c}{hostname}}{conj(paquete)}
  [$c$ $\in$ dameHostnames(computadoras(red($d$)))]
  {alias($res$ =$_{obs}$ enEspera($d$, dameCompu($c$)))}%
  [O(L)]
  [devuelve los paquetes en la cola de la computadora.]
  [res no es modificable.]
  
  \InterfazFuncion{IniciarDCNet}{\In{r}{red}}{dcnet}
  [true]
  {$res$ =$_{obs}$ iniciarDCNet($r$)}%
  [O(N*n*(L+n)]
  [crea una nueva Dcnet.]
  [la red se agrega por copia.]
    
  \InterfazFuncion{CrearPaquete}{\Inout{d}{dcnet}, \In{p}{paquete}}{}
  [$d$ = $d_0$ $\wedge$ $\neg$( ($\exists$ $p\sp{\prime}$: paquete) (paqueteEnTransito?($d$, $p\sp{\prime}$) $\wedge$ id($p\sp{\prime}$) = id($p$) ) $\wedge$ \newline
  origen($p$) $\in$ computadoras(red($d$)) $\wedge_L$ destino($p$) $\in$ computadoras(red($d$)) $\wedge_L$ hayCamino?(red($d$), origen($p$), destino($p$))]
  {$d$ $=_{obs}$ crearPaquete($d_0$, $p$)}%
  [O(L + log(k))]
  [agrega un paquete a la red.]
  [el paquete se agrega por copia.]

\newpage

  \InterfazFuncion{AvanzarSegundo}{\Inout{d}{dcnet}}{}
  [$d$ = $d_0$]
  {$d$ =$_{obs}$ avanzarSegundo($d_0$)}%
  [O(n * (L + log(k)))]
  [realiza los movimientos de paquetes correspondientes, aplicando los cambios necesarios a la dcnet.]
  []
  
  \InterfazFuncion{PaqueteEnTransito?}{\In{d}{dcnet}, \In{p}{IDpaquete}}{bool}
  [true]
  {$res$ =$_{obs}$ IDpaqueteEnTransito?($d$, $p$)}%
  [O(n * k)]
  [indica si el paquete esta en alguna de las colas dado el ID.]
  []
  
  \InterfazFuncion{LaQueMasEnvio}{\In{d}{dcnet}}{hostname}
  [true]
  {$res$ =$_{obs}$ laQueMasEnvio($d$).ip}%
  [O(1)]
  [devuelve la computadora que m\'as paquetes envi\'o.]
  [res se devuelve por copia.]  
  
  \InterfazFuncion{$\bullet = \bullet$}{\In{d_1}{dcnet}, \In{d_2}{dcnet}}{bool}
  [true]
  {$res$ =$_{obs}$ ($d_1$ =$_{obs}$ $d_2$}%
  [O(n*n * ( L + n*n + m + log(k) ) + n*(m*m + L))]
  [indica si dos dcnet son iguales.]
  []  
  
  \textbf{donde:} \newline  
   \TipoVariable{hostname} es \TipoVariable{string}, \newline
   \TipoVariable{interfaz} es \TipoVariable{nat}, \newline
   \TipoVariable{IDpaquete} es \TipoVariable{nat}, \newline
   \TipoVariable{compu} es \TipoVariable{tupla}< ip: \TipoVariable{hostname}, interfaces: \TipoVariable{conj}(\TipoVariable{interfaz})>, \newline
   \TipoVariable{paquete} es \TipoVariable{tupla}< id: \TipoVariable{IDpaquete}, prioridad: \TipoVariable{nat}, origen: \TipoVariable{hostname}, destino: \TipoVariable{hostname} >.
   
\end{Interfaz}

\newpage

\textbf{Especificaci\'on de las operaciones auxiliares utilizadas en la interfaz (no exportadas)}

\begin{tad}{\tadNombre{DCNet extendida}}

\tadExtiende{\tadNombre{DCNet}}

\tadOtrasOperaciones 

\tadOperacion{damehostnames}{conj(compu)}{conj(hostname)}{}
\tadOperacion{dameCompu}{dcnet /d, hostname /s}{compu}{$s$ $\in$ dameHostnames(computadoras(red($d$)))}
\tadOperacion{auxDameCompu}{hostname /s, conj(compu) /cc}{compu}{}
\tadOperacion{dameSecuDeHostnames}{secu(compu)}{secu(hostname)}{}
\tadOperacion{IDpaqueteEnTransito?}{dcnet /d, IDpaquete /p}{bool}{}
\tadOperacion{damePaquete}{dcnet /d, IDpaquete /p}{paquete}{IDpaqueteEnTransito?($d$,$p$)}
\tadOperacion{dameIDpaquetes}{conj(paquete)}{conj(IDpaquete)}{}

\tadAxiomas[\paratodo{dcnet}{d}, \paratodo{hostname}{s}, \paratodo{IDpaquete}{p}, \paratodo{conj(compu)}{cc}, , \paratodo{secu(compu)}{secu}, \paratodo{conj(paquete)}{cp}, ]

\tadAxioma{dameHostnames($cc$)}
{ \IF {vacio?($cc$)} 
THEN {$\emptyset$}
ELSE {Ag( ip(dameUno($cc$)), dameHostnames(sinUno($cc$)) )}
FI}

\tadAxioma{dameCompu($d$, $s$)}{auxDameCompu($s$, computadoras(red(($d$)))}

\tadAxioma{auxDameCompu($s$, $cc$)}
{ \IF {ip(dameUno($cc$)) = $s$} 
THEN {dameUno($cc$)}
ELSE {auxDameCompu($s$, sinUno($cc$))}
FI}

\tadAxioma{dameSecuDeHostnames($secu$)}
{ \IF {vacia?($secu$)} 
THEN {$<>$}
ELSE {ip(prim($secu$)) $\bullet$ dameSecuDeHostnames(fin($secu$)) }
FI}

\tadAxioma{IDpaqueteEnTransito?($d$, $p$)}{auxIDpaqueteEnTransito($d$, computadoras(red($d$)), $p$)}

\tadAxioma{auxIDpaqueteEnTransito($d$, $cc$, $p$)}
{ \IF {vacio?($cc$)} 
THEN {false}
ELSE { \IF {$p$ $\in$ dameIDpaquetes(enEspera(dameUno($cc$)))}
		THEN {true}
		ELSE { auxIDpaqueteEnTransito($d$, sinUno($cc$), $p$)}
		FI}
FI}

\tadAxioma{dameIDpaquetes($cp$)}
{ \IF {vacio?($cp$)} 
THEN {$\emptyset$}
ELSE {Ag( id(dameUno($cp$)), dameIDpaquetes(sinUno($cp$)) )}
FI}

\end{tad}

\newpage

\begin{Representacion}

\begin{Estructura}{dcnet}[estr\_dcnet]

\begin{Tupla}[estr\_dcnet]
	\tupItem{red}{red} \newline \nomoreitems
	\tupItem{computadoras}{dicc(hostname, X)} \newline \nomoreitems
	\tupItem{porHostname}{dicc$_{UNIV}$ (hostname, itDicc(hostname, X))} \newline \nomoreitems
	\tupItem{conMasEnvios}{itDicc(hostname, X)} \newline \nomoreitems
	\tupItem{caminos}{arreglo\_dimensionable de arreglo\_dimensionable de lista(hostname)} \nomoreitems
\end{Tupla}

\begin{Tupla}[X]
	\tupItem{indice}{nat} \newline \nomoreitems
	\tupItem{paquetes}{conj(paquete)} \newline \nomoreitems
	\tupItem{cola}{cola$_{LOG}$(itConj(paquete))} \newline \nomoreitems
	\tupItem{paqPorID}{dicc$_{LOG}$ (IDpaquete, itConj(paquete))} \newline \nomoreitems
	\tupItem{cantEnvios}{nat} \nomoreitems
\end{Tupla} 

\end{Estructura}

\Comment{El dicc$_{UNIV}$ (basado en un TRIE) nos permite acceder a un significado en O(L) con L el largo del hostname mas largo (utilizando como clave a los hostnames).}\newline
\Comment{Al guardar un iterador a la computadora con mas env\'ios podemos devolverla por aliasing en O(1), cumpliendo as\'i la complejidad pedida.}\newline
\Comment{Al utilizar una cola$_{LOG}$ (basada en un HEAP) podemos acceder al paquete con prioridad mas alta (el que se tiene que enviar) en O(1) y desencolarlo en O(log(n)) con n = cantidad de paquetes en la cola. Esto nos sirve para poder cumplir avanzarSegundo y nos mantiene dentro de lo pedido en crearPaquete.}\newline
\Comment{Al utilizar un dicc$_{LOG}$ (basado en AA-TREE) podemos acceder a un paquete por medio de su ID en O(log(n)) con n = cantidad de paquetes en la computadora (pudiendo borrarlo o modificarlo dentro de la misma complejidad). Esto nos sirve para cumplir caminoRecorrido, ya que podemos buscar un paquete en O(log(n)) dentro de cada computadora, adem\'as nos mantiene dentro de la complejidad pedida en crearPaquete y avanzarSegundo.}\newline
\Comment{Al tener cantEnvios nos permite obtener en O(1) la cantidad de env\'ios de cada computadora, lo que nos sirve para (en avanzarSegundo) poder calcular la computadora con m\'as env\'ios dentro de la complejidad y almacenarlo en conMasEnvios}\newline
\Comment{El \'indice nos sirve para (como comentamos antes) utilizarlo como posici\'on en el arreglo caminos y poder averiguar su camino m\'inimo en la complejidad pedida.}\newline
\Comment{Guardamos los paquetes en el conjunto paquetes para poder tenerlos en O(1) y cumplir con la complejidad de enEspera.} \newline



\comentario{REP en Castellano: \newline
1: Las compus de Red son las compus de DCNet.\newline
2: PorHostname y computadoras tienen el mismo conjunto de claves. \newline
3: PorHostname permite acceder a los datos de todas las computadoras a trav\'es de iteradores.\newline
4: Los indices de las computadoras van de 0 a n-1. \newline
5: Los indices no se repiten.\newline
6: ConMasEnvios es un interador a la computadora con mayor cant de envios.\newline
7: La matriz de caminos es de n x n.\newline
8: En la matriz caminos[i][j] se guarda uno de los caminos minimos de la red correspondiente al origen y destino correspondientes a los indices i, j, respectivamente. Si no hay, se guarda una lista vacia.\newline
9: Las claves del diccionario paquetesPorID son los ID del conjunto paquetes.\newline
10: El conjunto de paquetes y la cola de prioridad tienen el mismo tamano.\newline 
11: La cola ordena los paquetes por prioridad. (usando los observadores del TAD Cola de Prioridad Alaternativa adjunto). \newline
Para todos los paquetes de una computadora:\newline
12: El origen y el destino estan entre las computadoras de la dcnet.\newline
13: El origen y el destino son distintos.\newline
14: Hay un camino posible entre el origen y el destino.\newline
15: La computadora actual esta en el camino minimo entre el origen y el destino.\newline
16: El id es unico.\newline
17: Son accesibles por el dicc usando su ID.\newline
}

\tadAlinearFunciones{REP}{estr\_dcnet /d}

\Rep[estr\_dcnet]{
	\comentario{1} dameHostnames(computadoras($e$.red)) = claves ($e$.computadoras) $\wedge$ 
	\comentario{2} claves ($e$.computadoras) = claves ($e$.porHostname) $\wedge$ 
	\comentario{3} ($\forall$ $c$: hostname, $c$ $\in$ claves($e$.porHostname)) (
	SiguienteClave(obtener($e$.porHostname, $c$)) = $c$ $\wedge$
	SiguienteSignificado(obtener($e$.porHostname, $c$)) = obtener($e$.computadoras, $c$) ) $\wedge$ \newline
	($\forall$ $c$: hostname, $c$ $\in$ claves($e$.computadoras))
	\comentario{4} 0 $<$ obtener($e$.computadoras, $c$).indice $<$ $\#$claves($e$.computadoras)-1 $\wedge$ obtener($e$.computadoras, $c$).indice = ordenLexicografico($c$, claves($e$.computadoras)) $\wedge$
	\comentario{5} $\neg$($\exists$ $c\sp{\prime}$: hostname, $c\sp{\prime}$ $\in$ claves($e$.computadoras), $c$ $\neq$ $c\sp{\prime}$) obtener($e$.computadoras, $c\sp{\prime}$).indice = obtener($e$.computadoras, $c$).indice $\wedge$ 
	\comentario{6} $\neg$($\exists$ $c\sp{\prime}$: hostname, $c\sp{\prime}$ $\in$ claves($e$.computadoras), $c$ $\neq$ $c\sp{\prime}$) obtener($e$.computadoras, $c\sp{\prime}$).cantEnvios > SiguienteSignificado($e$.conMasEnvios).cantEnvios $\wedge$ 
	\comentario{7} tam($e$.caminos) = $\#$claves($e$.computadoras) $\wedge_L$ 
	($\forall$ $i$: nat, 0 $<$ i $<$ $\#$claves($e$.computadoras)-1 ) tam($e$.caminos[$i$]) = $\#$claves($e$.computadoras) $\wedge$ 
	\comentario{8} ($\forall$ $c1, c2$: hostname, $c1, c2$ $\in$ claves($e$.porHostname)) \newline
	$\neg$ $\emptyset$? (caminosMinimos($e$.red, dameCompu($c1$), dameCompu($c2$))) $\Rightarrow$ 
	$e$.caminos[obtener($e$.computadoras, $c1$).indice][obtener($e$.computadoras, $c2$).indice] = dameUno(caminosMinimos($e$.red, dameCompu($c1$), dameCompu($c2$))) $\wedge$ \newline
	$\emptyset$? (caminosMinimos($e$.red, dameCompu($c1$), dameCompu($c2$))) $\Rightarrow$ 
	$e$.caminos[obtener($e$.computadoras, $c1$).indice][obtener($e$.computadoras, $c2$).indice] = Vacia() $\wedge$ \newline
	($\forall$ $c$: hostname, $c$ $\in$ claves($e$.computadoras)) (
	\comentario{9} dameIDpaquetes(obtener($e$.computadoras, $c$).paquetes) = claves(obtener($e$.computadoras, $c$).paquetesPorID) $\wedge$ 
	\comentario{10} $\#$(obtener($e$.computadoras, $c$).paquetes) = $\#$(obtener($e$.computadoras, $c$).cola) $\wedge$ 
	\comentario{11} vacia?(obtener($e$.computadoras, $c$).cola) = $\emptyset$?(obtener($e$.computadoras, $c$).paquetes) $\wedge$ \newline Siguiente($\Pi_2$(proximo(obtener($e$.computadoras, $c$).cola))) $\in$ obtener($e$.computadoras, $c$).paquetes \newline $\wedge$ 
	$\neg$($\exists$ $p\sp{\prime}$: paquete, $p\sp{\prime}$ $\in$ obtener($e$.computadoras, $c$).paquetes) $p\sp{\prime}$.prioridad $<$ Siguiente($\Pi_2$(proximo(obtener($e$.computadoras, $c$).cola))).prioridad $\wedge$ \newline
	$\Pi_1$(proximo(obtener($e$.computadoras, $c$).cola)) = Siguiente($\Pi_2$(proximo(obtener($e$.computadoras, $c$).cola))).prioridad $\wedge$ \newline
	desencolar(obtener($e$.computadoras, $c$).cola) = armarCola(obtener($e$.computadoras, $c$).paquetes - $\{$Siguiente($\Pi_2$(proximo(obtener($e$.computadoras, $c$).cola)))$\}$ $\wedge$
	\newline
	($\forall$ $p$: paquete, $p$ $\in$ obtener($e$.computadoras, $c$).paquetes) (
	\comentario{12} origen($p$).ip $\in$ claves ($e$.computadoras) $\wedge$ 
destino($p$).ip $\in$ claves ($e$.computadoras) $\wedge$ 
	\comentario{13} origen($p$).ip $\neq$ destino($p$).ip $\wedge$ 
	\comentario{14} hayCamino?($e$.red, origen($p$), destino($p$)) $\wedge$ 
	\comentario{15} esta? ($c$, caminos[obtener($e$.computadoras, origen($p$).ip)][obtener($e$.computadoras, destino($p$).ip) ) $\wedge$
	\comentario{16} ($\forall$ $c\sp{\prime}$: hostname, $c\sp{\prime}$ $\in$ claves($e$.computadoras), $c\sp{\prime}$ $\neq$ $c$ )  $\neg$($\exists$ $p\sp{\prime}$: paquete, $p\sp{\prime}$ $\in$ obtener($e$.computadoras, $c\sp{\prime}$).paquetes, $p$ $\neq$ $p\sp{\prime}$) $p$.id = $p\sp{\prime}$.id 
	\comentario{17} 
	definido?(obtener($e$.computadoras, $c$).paquetesPorID, $p$.id) $\wedge_L$ \newline
	Siguiente(obtener(obtener($e$.computadoras, $c$).paquetesPorID,  $p$.id)) = $p$
}
\newpage


\textbf{Especificaci\'on de las funciones auxiliares utilizadas en Rep}

\tadAlinearFunciones{armarCola}{conj(paquetes)} 

\tadOperacion{armarCola}{conj(paquete)}{cola(paquete)}{}

\tadAxiomas[\paratodo{conj(paquete)}{cc}]

\tadAxioma{armarCola($cc$)}{
{ \IF {$\emptyset$?($cc$)} 
THEN {Vacia()}
ELSE {encolar(dameUno($cc$).prioridad, dameUno($cc$), armarCola(sinUno($cc$)))}
FI}
}

\textbf{}

\AbsFc[estr\_dcnet]{dcnet}[e]{ $d$ $|$ 
red($d$) = $e$.red $\wedge$ \newline
($\forall$ $c$: compu, $c$ $\in$ computadoras(red($d$))) ( \newline
cantidadEnviados($d$, $c$) = obtener($e$.computadoras, $c$.ip).cantEnvios $\wedge$ \newline
enEspera($d$, $c$) = obtener($e$.computadoras, $c$.ip).paquetes $\wedge$ \newline
($\forall$ $p$: paquete, $p$ $\in$  obtener($e$.computadoras, $c$.ip).paquetes ) 
caminoRecorrido ($d$, $p$) = $e$.caminos[obtener($e$.computadoras, origen($p$).ip).indice][obtener($e$.computadoras, $c$.ip).indice] ) 
}

\end{Representacion}

\begin{Algoritmos}

\begin{algorithm}
\caption{Implementaci\'on de Red}
\begin{algorithmic}[0]
\Function{iRed}{\In{d}{estr\_dcnet}}{$\rightarrow$ res: Red}
	\State res $\gets$ d.red \Comment{O(1)}
\EndFunction \Comment{O(1)}
\end{algorithmic}
\end{algorithm}

\begin{algorithm}
\caption{Implementaci\'on de CaminoRecorrido}
\begin{algorithmic}[0]
\Function{iCaminoRecorrido}{\In{d}{estr\_dcnet}, \In{p}{IDPaquete}}{$\rightarrow$ res: lista(hostname)}
	\State itCompu $\gets$ CrearIt(d.computadoras) \Comment{O(1)}
	\State yaEncontrado $\gets$ FALSE \Comment{O(1)}
	\While{HaySiguiente(itCompu) $\&\&$ $\neg$yaEncontrado} \Comment{Guarda: O(1)} \Comment{Se repite a lo sumo n veces} \Comment{O(n)}
		\If{Definido?(SiguienteSignificado(itCompu).paqPorID, p)} \Comment{O(log(k))} 
			\State paquete $\gets$ Significado(SiguienteSignificado(itCompu).paqPorID, p) \Comment{O(1)}
			\State yaEncontrado $\gets$ TRUE \Comment{O(1)}
		\Else
			\State Avanzar(itCompu) \Comment{O(1)}
		\EndIf
	\EndWhile
	
	\State res $\gets$ caminos[Significado(d.computadoras, $\pi$3(paquete)).indice][SiguienteSignificado(itCompu).indice] \Comment{O(1) + O(n) + O(1)}
\EndFunction \Comment{O( n * log(k) )}
\end{algorithmic}
\end{algorithm}

\begin{algorithm}
\caption{Implementaci\'on de paquetes enviados}
\begin{algorithmic}[0]
\Function{iCantidadEnviados}{\In{d}{estr\_dcnet}, \In{c}{hostname}}{$\rightarrow res$: $nat$}
	\State it $\gets$ Significado(d.porHostname, c) \Comment{O(L)}
	\State res $\gets$ SiguienteSignificado(it).cantEnvios\Comment {O(1)}
\EndFunction \Comment{O(L)}
\end{algorithmic}
\end{algorithm}

\begin{algorithm}
\caption{Implementaci\'on EnEspera}
\begin{algorithmic}[0]
\Function{iEnEspera}{\In{d}{estr\_dcnet}, \In{c}{hostname}}{$\rightarrow$ res: conj(paquete)}
	\State it $\gets$ Significado(d.porHostname, c) \Comment{O(L)}
	\State res $\gets$ SiguienteSignificado(it).paquetes \Comment{O(1)}
\EndFunction \Comment{O(L)}
\end{algorithmic}
\end{algorithm}


\begin{algorithm}
\caption{Implementaci\'on de iniciarDCNet}
\begin{algorithmic}[0]
\Function{iIniciarDCNet}{\In{r}{red}}{$\rightarrow res$: $estr\_dcnet$}
	\comentario{creo un diccionario lineal}
	\State diccCompus $\gets$ Vacio()  \Comment{O(1)}
	\comentario{creo un diccionario universal(trie)}
	\State diccHostname $\gets$ Vacio()  \Comment{O(1)}
\comentario{creo una lista vac\'ia donde voy a guardar los hostnames y ordenarlos}
	\State listaComp $\gets$ Vacia() \Comment{O(1)}
	\State itHostname $\gets$  CrearIt(Computadoras(r)) \Comment{O(1)}
	\State masEnvios $\gets$ Siguiente(itHostname) \Comment{O(1)}
	\While{HaySiguiente(itHostname)} \Comment{O(n) + O(1)}
\comentario{agrego el hostname a la lista de computadoras}
		\State AgregarAtras(listaComp, Siguiente(itHostname)) \Comment{O(L)}
		\comentario{Inicia el \'indice como cero, mas adelante les pondremos valor} 
		\State X $\gets$ <0, Vacio(), Vacio(), Vacio(), 0> \Comment{O(1) + O(1) + O(1) + O(1) + O(1)}
		\State itX $\gets$ DefinirRapido(diccCompus, Siguiente(itHostname), X)  \Comment{O(copy(hostname)) + O(copy(X))} 
		\State Definir(diccHostname, Siguiente(itHostname), itX)  \Comment{O(L) + O(copy(X))}
		\State Avanzar(itHostname) \Comment{O(1)}
	\EndWhile
	\State itPC $\gets$ CrearIt(diccCompus) \Comment{O(1)}
	\State itPC2 $\gets$ CrearIt(diccCompus) \Comment{O(1)}				 	 	
	\State n $\gets$ \#Claves(diccCompus) \Comment{O(1)}
	\State arrayCaminos $\gets$ CrearArreglo(n) \Comment{O(n)}
	\comentario{voy a crear un arreglo en cada posicion de arrayCaminos, el cual va a tener el minimo camino}
	\While{HaySiguiente(itPC)} \Comment{O($\#$Computadoras(r))}
		\State arrayDestinos $\gets$ CrearArreglo(n) \Comment{O(n)}
		\While{HaySiguiente(itPC2)} \Comment{O($\#$Computadoras(r))}
			\State ConjCaminos $\gets$ CaminosMinimos(r, SiguienteClave(itPC), SiguienteClave(itPC2)) \Comment{O(n*L)}
			\State itConj $\gets$ CrearIt(ConjCaminos) \Comment{O(1)}
			\comentario{de todos los caminos minimos me quedo con uno}
			\If{HaySiguiente(itConj)}   \Comment{O(1)}
				\State arrayDestinos[SiguienteSignificado(itPC2).indice] $\gets$ Siguiente(itConj) \Comment{O(1)}
			\Else 
				\comentario{si no hay camino, creo una lista vacia}
				\State arrayDestinos[SiguienteSignificado(itPC2).indice] $\gets$ Vacia() \Comment{O(1)}
			\EndIf
			
				\algstore {iniciar}
	
\end{algorithmic}
\end{algorithm}

\begin{algorithm}
\begin{algorithmic}[0]

	\algrestore {iniciar}
	
			\State Avanzar(itPC2) \Comment{O(1)}
		\EndWhile
		\State arrayCaminos[SiguienteSignificado(itPC).indice] $\gets$ arrayDestinos \Comment{O(1)}
		\State Avanzar(itPC) \Comment{O(1)}
	\EndWhile

\comentario{inicio el Indice en 0}
	\State indice $\gets$  0 \Comment{O(1)}	
	\While{indice < \#Claves(Computadoras(r))} \Comment{Guarda: O(n)} \Comment{se ejecuta n veces} \Comment{O(n)}
\comentario{busco el m\'inimo de la lista de hostnames (por \'orden alfab\'etico}
		\State itHostnames $\gets$ CrearIt(listaComp) \Comment{O(1)}
		\State min $\gets$ Copiar(Siguiente(itHostnames)) \Comment{O(L)}
		\State Avanzar(itHostnames) \Comment{O(1)}
		\While{HaySiguiente(itHostnames)}
			\If{min < Siguiente(itHostnames)} \Comment{O(L)}
				\State min $\gets$ Copiar(Siguiente(itHostnames)) \Comment{O(L)}
			\EndIf
			\State Avanzar(itHostnames) \Comment{O(1)}
		\EndWhile
		\State Significado(diccCompus, min).indice = indice \Comment{O(n)}
\comentario{creo un iterador de la lista para eliminar el minimo que ya use}
		\State itHostnames $\gets$ CrearIt(listaComp) \Comment{O(1)}
		\State noElimine $\gets$ TRUE \Comment{O(1)}
		\While{HaySiguiente(itHostnames) $\&\&$ noElimine} \Comment{se ejecuta a lo sumo n veces} \Comment{O(n)}
			\If{Siguiente(itHostnames) == min} \Comment{O(L)}
				\State EliminarSiguiente(itHostnames) \Comment{O(1)}
				\State noElimine $\gets$ FALSE \Comment{O(1)}
			\EndIf
			\State Avanzar(itHostnames) \Comment{O(1)}
		\EndWhile
		\State indice $\gets$ indice + 1 \Comment{O(1)}
	\EndWhile
	
	
	\State res $\gets$ < Copiar(r), diccCompus, diccHostname, masEnvios, arrayCaminos > \Comment{O(Copiar(r)) + O(1) + O(1) + O(1) + O(1)}
	
\EndFunction
\end{algorithmic}
\end{algorithm}

\begin{algorithm}
\caption{Implementaci\'on de crearPaquete}
\begin{algorithmic}[0]
\Function{iCrearPaquete}{\Inout{d}{estr\_dcnet}, \In{p}{paquete}}
	\State itPC $\gets$ Significado(d.porHostname, paquete.origen) \Comment{O(L)}
	\State itPaq $\gets$ AgregarRapido(SiguienteSignificado(itPC).paquetes, p) \Comment{O(copiar(nat) + O(L))} \Comment{O(L)}
	\State Encolar(SiguienteSignificado(itPC).cola, p.prioridad ,itPaq)	\Comment{O(log(n)), n cantidad de nodos)O(log(k)}
	\State Definir(SiguienteSignificado(itPC).paquetesPorID, IDpaquete, itPaq) \Comment{O(log(k))}
\EndFunction \Comment{O( L + log(k) )}
\end{algorithmic}
\end{algorithm}

\begin{algorithm}
\caption{Implementaci\'on de AvanzarSegundo}
\begin{algorithmic}[0]

\Function{iAvanzarSegundo}{inout d: estr\_dcnet}{}

	\State arreglo $\gets$ crearArreglo[$\#$Claves(d.computadoras)] de tupla(usado: bool, paquete: paquete, destino: string), donde paquete es tupla(IDpaquete: nat, prioridad: nat, origen: string, destino: string)
	\State \Comment {O(n) para calcular cantidad de claves, O(1) para crearlo}
	
	\For {(int i=0,  $<$ $\#$Claves(d.computadoras), i++)} \Comment {el ciclo se har\'a n veces}
	\State arreglo[i].usado = false \Comment {O(1)}
	\EndFor
	
	\comentario {Inicializo Iterador}
	\State itCompu $\gets$ crearIt(d.computadoras) \Comment {O(1)}
	\State i $\gets$ 0
	
	\State \Comment {Ciclo 1: Desencolo y guardo en arreglo auxiliar.}
	\While { (HaySiguiente(itCompu))} \Comment {el ciclo se har\'a a lo sumo n veces}
		\If {($\neg$(Vacia?(SiguienteSignificado(itCompu).cola)))} \Comment {O(1)}
			\comentario{Borro el de mayor priorirdad del heap:}
			\State itPaquete $\gets$ Desencolar(SiguienteSignificado(itCompu).cola)
			 \Comment {O(log k)}
			\comentario{Lo elimino del dicc AVL}
			\State Borrar(SiguienteSignificado(itCompu).paquetesPorID, Siguiente(itPaquete).IDpaquete)
			\State	\Comment {O(log k)}
			\comentario{Guardo el paquete en una variable}
			\State paqueteDesencolado $\gets$ Siguiente(itPaquete) \Comment {O(1)}
			\comentario{Lo elimino del conjunto lineal de paquetes}
			\State EliminarSiguiente(itPaquete) \Comment {O(1)}
			\comentario{Calculo proximo destino fijandome en la matriz}
			\comentario{El origen lo tengo en O(1) en el significado del iterador de compus.}
			\State origen $\gets$ (SiguienteSignificado(itCompu)).indice \Comment {O(1)}
			\comentario{El destino lo obtengo en O(L) buscando por hostname el destino del paquete, y luego guardo el indice.}
			\State itdestino $\gets$ Significado(d.porHostname, paqueteDesencolado.destino) 
				\Comment {O(L)}
			\State destino $\gets$ (SiguienteSignificado(itdestino)).indice 	\Comment {O(1)}
			\State proxDest $\gets$ d.caminos[origen][destino][1]	\Comment {O(1)}
			\comentario{Lo inserto en el arreglo junto con el destino s\'olo si el destino no era el final.}
			\If {(proxDest $\neq$ paqueteDesencolado.destino)}
				\State arreglo[i] $\gets$ $<$true, paqueteDesencolado, proxDest$>$ \Comment {O(1)}
			\EndIf
			\comentario{Aumento cantidad de env\'ios}
			\State SiguienteSignificado(itCompu).cantEnvios ++ \Comment {O(1)}
			\comentario{Actualizo conMasEnvios}
			\State envios $\gets$ SiguienteSignificado(itCompu).cantEnvios \Comment {O(1)}
			\If {(envios $>$ SiguienteSIgnificado(d.conMasEnvios).cantEnvios)} \Comment {O(1)}
				\State d.conMasEnvios $\gets$ itCompu
			\EndIf
		\EndIf
		\comentario{Avanzo de computadora}
		\State Avanzar(itCompu)	\Comment {O(1)}
		\State i++
	\EndWhile
	
	\algstore {ciclo2}
	
\end{algorithmic}
\end{algorithm}

\begin{algorithm}
\begin{algorithmic}[0]

	\algrestore {ciclo2}
	
	\State \Comment {Ciclo 2: Encolo los paquetes del vector a sus destinos correspondientes.}
	\State i $\gets$ 0
	\While { HaySiguiente(itCompu)} 	\Comment {el ciclo se har\'a a lo sumo n veces}
		\If {arreglo[i].usado}
			\comentario{Busco el proxDestino guardado en el arreglo por hostname.}
			\State itdestino $\gets$ Significado(d.porHostname, arreglo[i].destino) \Comment {O(L)}
			\comentario{Agrego el paquete al conjunto de paquetes del prox destino.}
			\State itpaquete $\gets$ AgregarRapido(SiguienteSignificado(itdestino).paquetes, arreglo[i].paquete) 
			\State \Comment {O(1)}
			\comentario{Encolo el heap del destino}
			\State prioridad $\gets$ (arreglo[i].paquete).prioridad
			\State Encolar(SiguienteSignificado(itdestino).cola, prioridad, itpaquete) \Comment {O(log k)}
			\comentario{Lo agrego en el dicc AVL.}
			\State IDpaq $\gets$ (arreglo[i].paquete).IDpaquete 		\Comment {O(1)}
			\State Definir(SiguienteSignificado(itdestino).paquetesPorID, IDpaq, itpaquete) \Comment {O(log k)}
		\EndIf
	\State i++
	\State Avanzar(itCompu)
	\EndWhile
	
\EndFunction \Comment{O( n * ( L + log(k) ) )}
\end{algorithmic}
\end{algorithm}

\begin{algorithm}
\caption{Implementaci\'on de PaqueteEnTransito?}
\begin{algorithmic}[0]
\Function{iPaqueteEnTransito?}{in d: estr\_dcnet, in p:IDpaquete}{$\rightarrow$ res: bool}
	\State res $\gets$ false \Comment{O(1)}
	\State itCompu $\gets$ crearIt(d.computadoras) \Comment{O(1)}
	\While { HaySiguiente(itCompu) $\&\&$ $\neg$res}  
		\newline  \Comment{a lo sumo n veces, la guarda es O(1)}
		\State itPaq $\gets$ crearIt(siguienteSignificado(itCompu).paquetes) \Comment{O(1)}
		\While {(HaySiguiente(itPaq) $\&\&$ Siguiente(itPaq).id $\neq$ p)} 
		\Comment{a lo sumo k veces, la guarda es O(1)}
			\State Avanzar(itPaq)  \Comment{O(1)}
		\EndWhile
		\If {Siguiente(itPaq) == p)} \Comment{O(1)}
			\State res $\gets$ True \Comment{O(1)}
		\EndIf
		\State Avanzar(itCompu) \Comment{O(1)}
	\EndWhile
\EndFunction \Comment{O(n * k)}
\end{algorithmic}
\end{algorithm}

\begin{algorithm}
\caption{Implementaci\'on de LaQueMasEnvi\'o}
\begin{algorithmic}[0]
\Function{iLaQueMasEnvi\'o}{in d: estr\_dcnet}{$\rightarrow$ res: hostname}
	\State res $\gets$ SiguienteClave(d.conMasEnvios)	 \Comment{O(1)}
\EndFunction \Comment{O(1)}
\end{algorithmic}
\end{algorithm}



\begin{algorithm}
\caption{Implementaci\'on de ==}
\begin{algorithmic}[0]
\Function{iIgualdad}{in d1: estr\_dcnet, in d2: estr\_dcnet}{$\rightarrow$ res: bool}
	\comentario {Comparo redes usando == de red}
	\State res $\gets$ (d1.red == d2.red)						\Comment {O(n*n * ( L + n*n + m ) + n*m*m)}
	\If {(res)} 													\Comment {O(1)}
		\State itCompu $\gets$ crearIt(d1.computadoras)  		\Comment {O(1)}
		\State string host 										\Comment {O(1)}
		\comentario {Recorro las computadoras}
		\While { (HaySiguiente(itCompu) $\&\&$ res) }	\Comment {itero O(n) veces, la guarda es O(1)}
			\State host $\gets$ SiguienteClave(itCompu)			 \Comment {O(1)}
			\comentario{Comparo enEspera usando == de conjunto lineal, y cant. enviados}
			\State res $\gets$ (enEspera(d1, host) == enEspera(d2, host) $\&\&$ \newline  cantidadEnviados(d1,host) == cantidadEnviados(d2,host))  \Comment {O(L)}
			\State itpaq $\gets$ crearIt(SiguienteSignificado(itCompu).paquetes) \Comment {O(1)}
			\State int j $\gets$ 0												\Comment {O(1)}
			\State nat id														\Comment {O(1)}
			\comentario{Recorro paquetes de cada computadora}
			\While { (HaySiguiente(itpaq) $\&\&$ res )}	\Comment {itero O(k) veces, la guarda es O(1)}
				\State id $\gets$ Siguiente(itpaq).IDpaquete 					\Comment {O(1)}
				\comentario {Comparo caminosRecorridos usando == de listas enlazadas}
				\State res $\gets$ (caminoRecorrido(d1, id) == caminoRecorrido(d2, id)) \Comment {O(n * log(k))}
				\State avanzar(itpaq)											\Comment {O(1)}
			\EndWhile
		\State avanzar (itCompu)													\Comment {O(1)}
		\EndWhile
	\EndIf
\EndFunction \Comment{O(n*n * ( L + n*n + m + log(k) ) + n*(m*m + L))}
\end{algorithmic}
\end{algorithm}

\end{Algoritmos}

\clearpage




\clearpage

\section{M�dulo AB}
\begin{Interfaz}
  
  %\textbf{usa}: \tadNombre{}.
  
  \textbf{se explica con}: \tadNombre{AB($\sigma$)}.
  
  \textbf{g\'eneros}: \TipoVariable{ab($\sigma$)}.

  \Titulos{Operaciones}

  \InterfazFuncion{nil?}{\In{a}{ab($\sigma$)}}{Bool}
  {res $\igobs$ nil?(a)}
  [O(1)]
  [Indica si el arbol est\'a vac\'io.]

  \InterfazFuncion{raiz}{\In{a}{ab($\sigma$)}}{$\sigma$}
  [$\neg$nil?(a)]
  {res $\igobs$ raiz(a)}
  [O(1)]
  [Devulve el valor de la raiz del arbol.]
  [El valor se devuelve por referencia, res es modificable.]
  
  \InterfazFuncion{izq}{\In{a}{ab($\sigma$}}{ab($\sigma$)}
  [$\neg$nil?(a)]  
  {res $\igobs$ vacio()}
  [O(1)]
  [Devulve el subárbol izquierdo.]
  [El subárbol se devuelve por referencia, res es modificable.]
  
  \InterfazFuncion{setearIzq}{\Inout{a}{ab($\sigma$}, \In{i}{ab($\sigma$}}{}
  [$\neg$nil?(a)]  
  {res $\igobs$ bin(i, raiz(a), der(a))}
  [O(1)]
  [Asigna i como nuevo subárbol izquierdo.]
  []
  
  \InterfazFuncion{der}{\In{a}{ab($\sigma$}}{ab($\sigma$)}
  [$\neg$nil?(a)]  
  {res $\igobs$ vacio()}
  [O(1)]
  [Devulve el subárbol derecho.]
  [El subárbol se devuelve por referencia, res es modificable.]
  
  \InterfazFuncion{setearDer}{\Inout{a}{ab($\sigma$}, \In{d}{ab($\sigma$}}{}
  [$\neg$nil?(a)]  
  {res $\igobs$ bin(izq(a), raiz(a), d)}
  [O(1)]
  [Asigna d como nuevo subárbol derecho.]
  []
  
  \InterfazFuncion{nil}{}{ab($\sigma$)}
  [true]
  {res $\igobs$ nil()}
  [O(1)]
  [Devuelve un \'arbol binario vacio.]
  
  \InterfazFuncion{bin}{\In{i}{ab($\sigma$)}, \In{d}{ab($\sigma$)}, \In{a}{$\sigma$}}{ab($\sigma$)}
  [true]
  {res $\igobs$ bin(i,d,a)}
  [O(copiar($\sigma$))]
  [Crea un \'arbol binario con raiz a, hijo derecho d e hijo izquierdo i.]

\end{Interfaz}

\begin{Representacion}

\begin{Estructura}{estr}[puntero(nodo($\sigma$))]
   
\begin{Tupla}[nodo($\alpha$)]
	\tupItem{izq}{ab($\sigma$)} \newline \nomoreitems
	\tupItem{der}{ab($\sigma$)} \newline \nomoreitems
	\tupItem{valor}{$\sigma$} \newline \nomoreitems
\end{Tupla} 

\end{Estructura}

\comentario{Rep en castellano:}
\comentario{1: Si un elemento es el hijo (izquierdo o derecho) de otro, entonces no es el hijo de ning\'un otro}
\comentario{2: Si un elemento es hijo izquierdo de cierto elemento, no puede ser tambi\'en el derecho}

\comentario{Nos referimos a estr.primero al primer nodo del arbol.}

Rep: estr $\rightarrow$ bool
Rep($estr$) $\equiv$ true $\iff$
\comentario{1} $((\forall n_1, n_2, n_3 : nodo(\alpha, \sigma)) ((n_1 \in arbol(estr.primero) \wedge\ n_2 \in arbol(estr.primero) \wedge\ n_3 \in arbol(estr.primero) \wedge\ (n_1 = n_2\rightarrow izq \vee\ n_1 = n_2\rightarrow der) \wedge\ n_2 \neq n_3) \Rightarrow_L\ (n_1 \neq n_3\rightarrow izq \wedge n_1 \neq n_3\rightarrow der))\ \wedge$
\comentario{2} $((\forall n_1, n_2 : nodo(\alpha, \sigma)) ((n_1 \in arbol(estr.primero) \wedge\ n_2 \in arbol(estr.primero) \wedge n_1 = n_2\rightarrow izq) \Rightarrow_L\ n_1 \neq n_2\rightarrow der)$

\AbsFc[estr]{ab($\sigma$)}[e]{$a$ $|$ 
nil?($a$) = (e == NULL) \newline
raiz($a$) = e-> valor \newline
izq($a$) = e->izq \newline
der($a$) e->der}

\textbf{Especificaci\'on de las operaciones auxiliares utilizadas para Rep y Abs}
\tadOperacion{arbol}{nodo($\alpha$, $\sigma$)}{conj(nodo($\alpha$, $\sigma$))}{}
\tadOperacion{caminoHastaRaiz}{nodo($\alpha$, $\sigma$)}{nat}{}
\tadAxioma{arbol(n)}{
\IF {$n.izq \neq null \wedge n.der \neq null$}
THEN {Ag($n,arbol(n.izq)\cup arbol(n.der)$)}
ELSE { \IF {$n.izq \neq null$}
		THEN {Ag($n,arbol(n.izq)$)}
		ELSE { \LIF\ $n.der \neq null$ \LTHEN\ Ag($n,arbol(n.der)$) \LELSE\ Ag($n,\emptyset$) \LFI}
		FI}
FI}

\tadAxioma{caminoHastaRaiz(n)}{\textbf{if} $n.padre = null$ \textbf{then} $0$ \textbf{else} $caminoHastaRaiz(n.padre)+1$ \textbf{fi}}

\end{Representacion}

\begin{Algoritmos}

\begin{algorithm}
\caption{Implementaci\'on de nil?}
\begin{algorithmic}[0]
\Function{iNil?}{in a: ab($\sigma$)}{$\rightarrow$ res: bool}
	\State res $\gets$ a == NULO) \Comment{O(1)}
\EndFunction
\end{algorithmic}
\end{algorithm}

\begin{algorithm}
\caption{Implementaci\'on de raiz}
\begin{algorithmic}[0]
\Function{iRaiz}{in a: ab($\sigma$)}{$\rightarrow$ res: $\sigma$}
	\State res $\gets$ a$\rightarrow$raiz \Comment{O(1)}
\EndFunction
\end{algorithmic}
\end{algorithm}

\begin{algorithm}
\caption{Implementaci\'on de izq}
\begin{algorithmic}[0]
\Function{iIzq}{in a: ab($\sigma$)}{$\rightarrow$ res: ab($\sigma$)}
	\State res $\gets$ a$\rightarrow$izq \Comment{O(1)}
\EndFunction
\end{algorithmic}
\end{algorithm}

\begin{algorithm}
\caption{Implementaci\'on de setearIzq}
\begin{algorithmic}[0]
\Function{iSetearIzq}{inout a: ab($\sigma$), in i:ab($\sigma$)}{}
	\State a$\rightarrow$izq $\gets$ i \Comment{O(1)}
\EndFunction
\end{algorithmic}
\end{algorithm}

\begin{algorithm}
\caption{Implementaci\'on de der}
\begin{algorithmic}[0]
\Function{iDer}{in a: ab($\sigma$)}{$\rightarrow$ res: ab($\sigma$)}
	\State res $\gets$ a$\rightarrow$der \Comment{O(1)}
\EndFunction
\end{algorithmic}
\end{algorithm}

\begin{algorithm}
\caption{Implementaci\'on de setearDer}
\begin{algorithmic}[0]
\Function{iSetearDer}{inout a: ab($\sigma$), in d:ab($\sigma$)}{}
	\State a$\rightarrow$der $\gets$ d \Comment{O(1)}
\EndFunction
\end{algorithmic}
\end{algorithm}

\begin{algorithm}
\caption{Implementaci\'on de nil}
\begin{algorithmic}[0]
\Function{iNil}{}{$\rightarrow$ res: ab($\sigma$)}
	\State res $\gets$ NULL \Comment{O(1)}
\EndFunction
\end{algorithmic}
\end{algorithm}

\begin{algorithm}
\caption{Implementaci\'on de bin}
\begin{algorithmic}[0]
\Function{iBin}{in i: ab($\sigma$), in a: $\sigma$, in d: ab($\sigma$)}{$\rightarrow$ res: ab($\sigma$)}
	\State res$\rightarrow$izq $\gets$ i \Comment{O(1)}
	\State res$\rightarrow$der $\gets$ d \Comment{O(1)}
	\State res$\rightarrow$raiz $\gets$ a \Comment{O(copiar($\sigma$))}
\EndFunction
\end{algorithmic}
\end{algorithm}

\end{Algoritmos}

\clearpage

\section{M�dulo Cola de Prioridad Logaritmica ($\alpha$) }
\begin{Interfaz}

  \textbf{usa}: \tadNombre{tupla,  nat, bool, $\alpha$}.
  
  \textbf{se explica con}: \tadNombre{Cola de Prioridad Alternativa}.

  \textbf{g\'eneros}: \TipoVariable{colaLog($\alpha$)}.

  \Titulos{Operaciones de Cola de Prioridad $_{HEAP}$}

  \InterfazFuncion{Vacia}{}{colaLog($\alpha$)}
  {res =$_{obs}$ vacia}%
  [$O(1)$]
  [Crea una cola vacia.]
  []

  \InterfazFuncion{Vacia?}{\In{estr}{colaLog($\alpha$)}}{bool}
  {$res$ =$_{obs}$ vacia?($estr$)}%
  [$O(1)$]
  [Indica si la cola esta vacia.]
  []

  \InterfazFuncion{Pr\'oximo}{\In{estr}{colaLog($\alpha$)}}{tupla($nat, \alpha$)}
  [$\neg Vacia?(estr)$]
  {$res =_{obs} proximo(estr)$}%
  [$O(copiar(\alpha))$]
  [Devuelve una tupla que contiene al pr\'oximo elemento y su prioridad.]
  

  \InterfazFuncion{Encolar}{\Inout{estr}{colaLog($\alpha$)}, \In{prio}{nat}, \In{valor}{$\alpha$}}{bool} %res no me sirve para nada...lo puedo sacar o siempre tiene que estar aunque lo ponga siempre en true?
  [estr = estr$_0$]
  {$res$ $\wedge$ estr =$_{obs}$ encolar(estr$_0$)}%
  [$O(log(n) + copiar(\alpha))$]
  [Crea un nuevo elemento con los parametros dados y lo agrega a la cola.]
  []

  \InterfazFuncion{Desencolar}{\Inout{estr}{colaLog($\alpha$)}}{$\alpha$}
  [estr = $estr_0 \ \wedge \neg Vacia?(estr)$] 
  {estr =$_{obs}$ desencolar($estr_0$) $\wedge \ res =_{obs} proximo(estr_0)$}%
  [$O(log(n) + copiar(\alpha) + borrar(\alpha))$]
  [Devuelve al elemento de mayor prioridad y lo remueve de la cola. La cola no debe estar vac\'ia.]
  []
  
  
\end{Interfaz}

\newpage

\begin{tad}{\tadNombre{Cola de prioridad alternativa($\alpha$)}}
\tadGeneros{colaPrio($\alpha$)}
\tadExporta{colaPrio($\alpha$), generadores, observadores}
\tadUsa{\tadNombre{Bool, Nat, Tupla}}

\tadAlinearFunciones{desencolar}{nat,$\alpha$,colaPrior($\alpha$)}

\tadObservadores
\tadOperacion{vac\'ia?}{colaPrior($\alpha$)}{bool}{}
\tadOperacion{pr\'oximo}{colaPrior($\alpha$)/c}{tupla($nat, \alpha$)}{$\neg$ vac\'ia?($c$)}
\tadOperacion{desencolar}{colaPrior($\alpha$)/c}{colaPrior($\alpha$)}{$\neg$ vac\'ia?($c$)}

\tadGeneradores
\tadOperacion{vac\'ia}{}{colaPrior($\alpha$)}{}
\tadOperacion{encolar}{nat,$\alpha$,colaPrior($\alpha$)}{colaPrior($\alpha$)}{}

\tadAxiomas[\paratodo{colaPrior($\alpha$)}{c}, \paratodo{$\alpha$}{e}]
\tadAlinearAxiomas{desencolar(encolar($p$, $e$, $c$))}{}

\tadAxioma{vac\'ia?(vac\'ia)}{true}
\tadAxioma{vac\'ia?(encolar($p$, $e$, $c$))}{false}

\tadAxioma{pr\'oximo(encolar($p$, $e$, $c$))}{\textbf{if} vac\'ia?($c$) $\oluego$\ $\Pi_1(proximo(c)) < p$ \textbf{then} $<p,e>$ \textbf{else} \textbf{if} $\Pi_1(proximo(c)) = p$ \textbf{then} $<p,e>  \vee \ proximo(c)$ \textbf{else} $proximo(c)$ \textbf{fi} \textbf{fi}}
\tadAxioma{desencolar(encolar($p$, $e$, $c$))}{\textbf{if} vac\'ia?($c$) $\oluego$\ $\Pi_1(proximo(c)) < p$ \textbf{then} $c$ \textbf{else} \textbf{if} $\Pi_1(proximo(c)) = p$ \textbf{then} $c\  \vee \ encolar(p, e, desencolar(c)$ \textbf{else}  $encolar(p, e, desencolar(c)$ \textbf{fi} \textbf{fi}}

\end{tad}

\newpage

\begin{Representacion}

\begin{Estructura}{colaLog($\alpha$)}[estr\_heap($\alpha$)]

\begin{Tupla}[estr\_heap($\alpha$)]
	\tupItem{size}{nat} \newline \nomoreitems
	\tupItem{arb}{$ab$(tupla($prio$: nat, $valor$: $\alpha$, $padre$: ab($\alpha$)))} \newline \nomoreitems
\end{Tupla}

\comentario{De aqu\'i en adelante, para el rep y abs, por cuestiones de brevedad y legibilidad, se referir\'a al primer nodo del arbol como estr.primero; al padre de cada nodo simplemente como nodo.padre; a el valor propiamente dicho como nodo.valor y a cada nodo(tupla(nat, tupla(puntero(nodo$\alpha$), $\alpha$))) como nodo($\alpha$).}

\end{Estructura}

\comentario{Rep en castellano:}
\comentario{El invariante del \'arbol binario se sigue cumpliendo para arb}
\comentario{1: El tama\~no del \'arbol (size) debe ser igual a la cantidad de nodos en el \'arbol}
\comentario{2: El primer elemento no tiene padre}
\comentario{3: Todos los nodos, con la excepci\'on del primero, deben tener padre, y deben ser el hijo izquierdo o (exclusivo) derecho de su padre (el invariante del \'arbol binario garantiza que ning\'un nodo pueda ser tanto el hijo izquierdo como el derecho de su padre)}
\comentario{4: La prioridad del padre es menor o igual a la de los hijos}
\comentario{5: La altura del \'arbol es igual a uno m\'as la parte entera del logaritmo de su tama\~no (es decir que la altura del \'arbol es O(log(n))}

Rep: estr\_heap($\alpha$) $\rightarrow$ bool
\newline \indent Rep($estr$) $\equiv$ true $\iff$
\comentario{1} $size = \# arbol(estr.primero) \ \wedge_L$
\comentario{2} $((estr.primero).padre = null \wedge$
\comentario{3} $(\forall n: nodo(\alpha))( n \in arbol(estr.primero) \wedge \ n \neq estr.primero \Rightarrow (n.padre \neq null \wedge_L (((n.padre).izq = n \vee (n.padre).der = n ) ))) \wedge$
\comentario{4} $(\forall n: nodo(\alpha))( n \in arbol(estr.primero) \Rightarrow ((n.izq \neq null \Rightarrow n.prio \leq (n.izq).prio) \wedge (n.der \neq null \Rightarrow n.prio \leq (n.der).prio))) \wedge$
\comentario{5} $(\forall n: nodo(\alpha))( n \in arbol(estr.primero) \Rightarrow caminoHastaRaiz(n,arbol(estr.primero)) \leq \left \lfloor{log_2(size)}\right \rfloor + 1))$

\textbf{}
\textbf{}

Abs: estr\_heap($\alpha$) $e \rightarrow$ colaPrio($\alpha$) \{ Rep($e$) \} \newline
Abs($e$) $\equiv$ c: colaPrio($\alpha$) | $(Vacia?(c) \iff e.primero == NULO) \wedge_L
\newline \indent (\neg Vacia?(c) \Rightarrow Proximo(c) =\ <*(estr.primero)).prioridad, *(estr.primero)).valor> \ \wedge
\newline \indent (\neg Vacia?(e) \Rightarrow Desencolar(e) = Desencolar(c))$

\textbf{}
\textbf{}

\comentario{Las operaciones auxiliares utilizadas en la especificaci\'on del Rep y el Abs est\'an detalladas en la Estructura del Arbol Binario}	
%\textbf{if} $n.izq \neq null \wedge n.der \neq null$ \textbf{then} Ag($n.valor,arbol(n.izq)\cup arbol(n.der)$) \textbf{else}
%\textbf{if} $n.izq \neq null$ \textbf{then} Ag($n.valor,arbol(n.izq)$) \textbf{else} \textbf{if} $n.der \neq null$ \textbf{then} Ag($n.valor,arbol(n.der)$) \textbf{else} Ag($n.valor,\emptyset$)}


\end{Representacion}

\newpage

\begin{Algoritmos}

\begin{algorithm}
\caption{Implementaci\'on de Vacia}
\begin{algorithmic}[0]
\Function{iVacia}{}{$\rightarrow$ res: colaLog($\alpha$)}
\State res $\gets$ <0, nil()> \Comment{O(1)}
\EndFunction
\end{algorithmic}
\end{algorithm}

\begin{algorithm}
\caption{Implementaci\'on de Vacia?}
\begin{algorithmic}[0]
\Function{iVacia?}{\In{estr}{estr\_heap($\alpha$)}}{$\rightarrow res$: bool}
\State res $\gets$ nil?(estr.arb) \Comment{O(1)}
\EndFunction
\end{algorithmic}
\end{algorithm}

\begin{algorithm}
\caption{Implementaci\'on de Pr\'oximo}
\begin{algorithmic}[0]
\Function{Pr\'oximo}{\In{estr}{estr\_heap($\alpha$)}}{$\rightarrow res$: tupla($nat, \alpha$)}
\State res $\gets$ <(raiz(estr.arb)).prio, (raiz(estr.arb)).valor> \Comment{O(copiar($\alpha$))}
\EndFunction
\end{algorithmic}
\end{algorithm}

\begin{algorithm}
\caption{Implementaci\'on de Encolar}
\begin{algorithmic}[0]
\Function{iEncolar}{\Inout{estr}{estr\_heap($\alpha$)}, \In{prio}{nat}, \In{valor}{$\alpha$}}{$\rightarrow res$: bool}
	\State res $\gets$ true	 \Comment{O(1)}
	\If{nil?(estr.arb)} \Comment{O(1)}
		\State $estr.ab \gets$ bin(nil(), <prio, valor, nil()>, nil()) \Comment{O(copiar($\alpha$))}
	\Else
		\State size++ \Comment{O(1)}
		\State x $\gets$ size \Comment{O(1)}
		\State y $\gets$ <> \Comment{O(1)}
		\While{x $\neq$ 0} \Comment{La cantidad de veces que se ejecuta el ciclo es igual a la altura del heap. Al ser un arbol binario completo, la altura siempre ser\'a O(log(n))}
			\State y $\gets$ (x \% 2) $\bullet$ y \Comment{O(1)}
			\State x $\gets$ x $\/$ 2 \Comment{O(1)}
		\EndWhile
		\State y $\gets$ com(y) \Comment{O(log(n))}
		\State z $\gets$ estr.arb \Comment{O(1)}
		\State y $\gets$ fin(y) \Comment{O(1)}
		\While{long(y)>1} \Comment{El ciclo se ejecuta O(log(n)) veces}
			\State z $\gets$ \textbf{if} prim(y) $==$ 0 \textbf{then} izq(z) \textbf{else} der(z) \Comment{O(1)}
			\State y $\gets$ fin(y) \Comment{O(1)}
		\EndWhile
		\State w $\gets$ bin(nil(), <prio,valor, z>, nil()) \Comment{O(copiar($\alpha$))}
		\If{prim($y$) $==$ 0} \Comment{O(1)}
			\State setearIzq(z, w) \Comment{O(1)}
		\Else
			\State setearDer(z, w) \Comment{O(1)}
		\EndIf
		\While{w $\neq$ estr.abs $\wedge_L$ $\pi_1$(raiz(w)) > $\pi_1$(raiz($\pi_3$(raiz(w))))} \Comment{La cantidad de veces que se ejecuta el ciclo es a lo sumo la altura del heap, que es O(log(n))}
			\State aux $\gets$ $\pi_1$(raiz(w)) \Comment{O(copiar($\alpha$))}
			\State $\pi_2$(raiz(w)) $\gets$ $\pi_2$(raiz($\pi_3$(raiz(w)))) \Comment{O(copiar($\alpha$))}
			\State $\pi_2$(raiz($\pi_3$(raiz(w)))) $\gets$ aux \Comment{O(copiar($\alpha$))}
			\State aux2 $\gets$ $\pi_1$(raiz(w)) \Comment{O(1)}
			\State $\pi_1$(raiz(w)) $\gets$ $\pi_1$(raiz($\pi_3$(raiz(w)))) \Comment{O(1)}
			\State $\pi_1$(raiz($\pi_3$(raiz(w)))) $\gets$ aux2 \Comment{O(1)}
			\State w $\gets$ $\pi_3$(raiz(w)) \Comment{O(1)}
		\EndWhile
	\EndIf
\EndFunction
\end{algorithmic}
\end{algorithm}


\begin{algorithm}
\caption{Implementaci\'on de Desencolar}
\begin{algorithmic}[0]
\Function{iDesencolar}{\Inout{estr}{estr\_heap($\alpha$)}}{$\rightarrow res$: $\alpha$}
	\State res $\gets$  raiz(estr.arb) \Comment{O(1)}
	\State x $\gets$  size \Comment{O(1)}
	\State y $\gets$   <> \Comment{O(1)}
	\While{x $\neq$ 0} \Comment{La cantidad de veces que se ejecuta el ciclo es igual a la altura del heap. Al ser un arbol binario completo, la altura siempre ser\'a O(log(n))}
		\State y $\gets$ (x \% 2) $\bullet$ y \Comment{O(1)}
		\State x $\gets$ x $\/$ 2 \Comment{O(1)}
	\EndWhile
	\State y $\gets$ com(y) \Comment{O(log(n))}
	\State z $\gets$ estr.arb \Comment{O(1)}
	\State y $\gets$ fin(y) \Comment{O(1)}
	\While{long(y) >1} \Comment{El ciclo se ejecuta O(log(n)) veces}
		\State z $\gets$ \textbf{if} prim(y) $==$ 0 \textbf{then} izq(z) \textbf{else} der(z) \Comment{O(1)}
		\State y $\gets$ fin(y) \Comment{O(1)}
	\EndWhile
	\State w $\gets$ bin(nil() ,<prio, valor, nil()>, nil()> \Comment{O(copiar($\alpha$))}
	\State raiz(w).padre $\gets$ z \Comment{O(1)}
	\If{prim(y) $==$ 0} \Comment{O(1)}
		\State setearIzq(z, w) \Comment{O(1)}
	\Else
		\State setearDer(z, w) \Comment{O(1)}
	\EndIf
	\State (raiz(estr.arb)).valor $\gets$ raiz(z).valor \Comment{O(copiar($\alpha$))}
	\State (raiz(estr.arb)).prio $\gets$ raiz(z).prio \Comment{O(1)}
	\State borrar(z) \Comment{O(borrar($\alpha$))}
	\State z $\gets$ estr.arb \Comment{O(1)}
	\State size-- \Comment{O(1)}
	\While{(izq(z) $\neq$ null  $\vee$ der(z) $\neq$ null) $\wedge_L$  raiz(z).prio < minPrio(izq(z), der(z))} \Comment{La cantidad de veces que se ejecuta el ciclo es a lo sumo la altura del heap, que es O(log(n))}

	\State \comentario{minPrio devuelve la m\'inima prioridad entre los nodos si ambos punteros son validos, o la prioridad apuntada por
	el puntero no nulo en caso de que alguno no lo sea}
	\If{der(z) $==$ null  $\vee_L$  raiz(izq(z)).prio $\geq$ raiz(der(z)).prio} \Comment{O(1)}
		\State aux $\gets$ raiz(z).valor \Comment{O(copiar($\alpha$))}
		\State raiz(z).valor $\gets$ raiz(izq(z)).valor \Comment{O(copiar($\alpha$))}
		\State raiz(izq(z)).valor $\gets$ aux \Comment{O(copiar($\alpha$))}
		\State aux2 $\gets$ raiz(z).prio \Comment{O(1)}
		\State raiz(z).prio $\gets$ raiz(izq(z)).prio \Comment{O(1)}
		\State raiz(izq(z)).prio $\gets$ aux2 \Comment{O(1)}
		\State z $\gets$ izq(z) \Comment{O(1)}
	\Else
		\State aux $\gets$ raiz(z).valor \Comment{O(copiar($\alpha$))}
		\State raiz(z).valor $\gets$ raiz(der(z)).valor \Comment{O(copiar($\alpha$))}
		\State raiz(der(z)).valor $\gets$ aux \Comment{O(copiar($\alpha$))}
		\State aux2 $\gets$  raiz(z).prio \Comment{O(1)}
		\State raiz(z).prio $\gets$ raiz(der(z)).prio \Comment{O(1)}
		\State raiz(der(z)).prio $\gets$ aux2 \Comment{O(1)}
		\State z $\gets$ der(z) \Comment{O(1)}
	\EndIf
	\EndWhile


\EndFunction
\end{algorithmic}
\end{algorithm}

\FloatBarrier
\end{Algoritmos}




\clearpage

\section{M�dulo Diccionario Universal ($\sigma$) }
\begin{Interfaz}
  
  %\textbf{usa}: \tadNombre{}.
  
  \textbf{se explica con}: \tadNombre{Diccionario(STRING, $\sigma$)}.
  
  \textbf{g\'eneros}: \TipoVariable{diccUniv($\kappa, \sigma$)}.

  \Titulos{Operaciones}

  \InterfazFuncion{Definida}{\In{d}{diccUniv(STRING, $\sigma$)}, \In{c}{STRING}}{Bool}
  {res $\igobs$ def?(c,d)}
  [O(L), donde L es la cantidad de caracteres de la clave m\'as grande.]
  [Indica si la clave dada est\'a definida en el diccionario.]

  \InterfazFuncion{Significado}{\In{d}{diccUniv(STRING, $\sigma$)}, \In{c}{STRING}}{$\sigma$}
  [def?(c,d)]
  {res $\igobs$ obtener(c,d)}
  [O(L)]
  [Devuelve el significado asociado a la clave dada.]
  [Devuelve al significado por alias.]
  
  \InterfazFuncion{vacio}{}{diccUniv(STRING,$\sigma$)}
  {res $\igobs$ vacio()}
  [O(1)]
  [Crea un diccionario vac\'io.]
  
  \InterfazFuncion{definir}{\Inout{d}{diccUniv(STRING,$\sigma$)}, \In{c}{}STRING, \In{s}{$\sigma$}}{Bool}
  [$\neg$def?(c,d) $\wedge$ d=d$_0$]
  {$d = definir(c,d_0)$}
  [$O(L)+O(copiar(\sigma)$]
  [Agrega la clave al diccionario, asoci\'andole el significado dado como par\'ametro. $res$ indica si la clave ya estaba definida.]
  
  \InterfazFuncion{borrar}{\Inout{d}{diccUniv(STRING,$\sigma$)}, \In{c}{STRING}}{Bool}
  [d=d$_0$]
  {d=borrar(c,d$_0$)}
  [$O(L)+O(borrar(\sigma)$]
  [Borra la clave dada y su significado del diccionario. $res$ indica si la clave estaba definida (su valor es $true$ en caso de estarlo).]
  
  \InterfazFuncion{claves}{\In{d}{diccUniv(STRING,$\sigma$)}}{conj(STRING)}
  {res $\igobs$ claves(d)}
  [O(n*L), donde n es la cantidad de claves.]
  [Devuelve el conjunto de las claves del diccionario.] 

\comentario {Dise\~{n}o provisto por la c\'atedra.}

\end{Interfaz}


\clearpage

\section{M�dulo Diccionario Logaritmico ($\kappa$, $\sigma$) }
\begin{Interfaz}
  
  \textbf{usa}: \tadNombre{Bool, Nat}.
  
  \textbf{se explica con}: \tadNombre{Diccionario($\kappa, \sigma$)}.

  \textbf{g\'eneros}: \TipoVariable{diccLog($\kappa, \sigma$)}.

  \Titulos{Operaciones}

  \InterfazFuncion{Definido?}{\In{d}{diccLog($\kappa, \sigma$)}, \In{c}{$\kappa$}}{Bool}
  {res $\igobs$ def?(c,d)}%
  [$O(log(n) * comparar(\kappa))$]

  \InterfazFuncion{Significado}{\In{d}{diccLog($\kappa, \sigma$)}, \In{c}{$\kappa$}}{$\sigma$}
  [def?(c,d)]
  {res $\igobs$ obtener(c,d)}%
  [$O(log(n) * comparar(\kappa) + copiar(\sigma)$]
  
  \InterfazFuncion{Vacio}{}{diccLog($\kappa, \sigma$)}
  {res $\igobs$ vacio()}%
  [$O(1)$]
  
  \InterfazFuncion{Definir}{\Inout{d}{diccLog($\kappa, \sigma$)}, \In{c}{$\kappa$}, \In{s}{$\sigma$}}{}
  [$\neg$def?(c,d) $\wedge$ d=d$_0$]
  {d=definir(c,d$_0$)}%
  [O(log(n))]
  
  \InterfazFuncion{Borrar}{\Inout{d}{diccLog($\kappa, \sigma$)}, \In{c}{$\kappa$}}{}
  [def?(c,d) $\wedge$ d=d$_0$]
  {d=borrar(c,d$_0$)}%
  [$O(log(n) * comparar(\kappa) + max(borrar(\kappa), borrar(\sigma)) + max(copiar(\kappa), copiar(\sigma)))$]

\end{Interfaz}

\newpage 

\begin{Representacion}

\begin{Estructura}{diccLog}[$estr$: $ab$($clave$: $\kappa$, $nivel$: nat, $significado$: $\sigma$)]
 
\comentario{De aqu\'i en adelante, para el rep y abs, por cuestiones de brevedad y legibilidad se referir\'a al primer elemento del \'arbol binario (arb.primero) simplemente como primero; al nivel de cada nodo (nodo.prio.nivel) como nodo.nivel; a la clave de cada nodo (nodo.prio.clave) como nodo.clave; al significado de cada nodo (nodo.valor) como nodo.significado; a cada nodo(tupla($\kappa$, nat), $\sigma$) como nodo($\kappa$, $\sigma$.}

\comentario{La estructura utilizada para representar al diccionario Logaritmico es un AA tree. Es un tipo de ABB auto-balanceado que provee busqueda, insercion y borrado
en tiempo logaritmico. Los AA trees son similares a los Red-Black Trees, pero solo pueden tener hijos derechos ``rojos'' (en vez de utilizar un valor booleano de color, usan un valor entero de nivel;
los hijos ``rojos'' son los que tienen mismo nivel que sus padres), lo que reduce considerablemente la cantidad de operaciones necesarias para mantener el arbol.}
\comentario{http://user.it.uu.se/$\sim$arnea/abs/simp.html}
% http://user.it.uu.se/~arnea/abs/simp.html
   
\end{Estructura}

\comentario{Rep en castellano:}
\comentario{Se sigue cumpliendo el invariante del \'Arbol Binario}
\comentario{1: Las claves de todos los elementos del sub-\'arbol izquierdo de un nodo son menores que la suya. Las claves de todos los elementos del sub-\'arbol derecho de un nodo son mayores que la suya.}
\comentario{2: El nivel de toda hoja es 1.}
\comentario{3: El nivel de cada hijo izquierdo es exactamente 1 menos que el de su padre.}
\comentario{4: El nivel de cada hijo derecho es igual o 1 menos que el de su padre.}
\comentario{5: El nivel de cada nieto derecho (hijo derecho del hijo derecho de un nodo) es estrictamente menor que el de su abuelo.}
\comentario{6: Cada nodo con nivel (estrictamente) mayor a 1 tiene dos hijos.}


$\ $ \newline $\ $
\par Rep: estr $\rightarrow$ bool
\newline \indent Rep($e$) $\equiv$ true $\iff$ 
\comentario{1} $((\forall n_1, n_2:\ nodo(\kappa, \sigma))(\ n_1 \in\ arbol(e) \wedge \ n_2 \in\ arbol(e) \Rightarrow_L\ (n_1 \in arbol(n_2.izquierdo) \Rightarrow n_1.clave < n_2.clave) \wedge\ (n_1 \in arbol(n_2.derecho) \Rightarrow n_1.clave > n_2.clave) \wedge$
\comentario{2} $((\forall n:\ nodo(\kappa, \sigma))(\ n \in\ arbol(e) \wedge n.izquierdo = NULO \wedge n.derecho = NULO \Rightarrow\ n.nivel = 1) \wedge$
\comentario{3} $((\forall n:\ nodo(\kappa, \sigma))(\ n \in\ arbol(e) \wedge n.izquierdo \neq NULO \Rightarrow\ (n.izquierdo).nivel = n.nivel-1) \wedge$
\comentario{4} $((\forall n:\ nodo(\kappa, \sigma))(\ n \in\ arbol(e) \wedge n.derecho \neq NULO \Rightarrow ((n.derecho).nivel = n.nivel-1 \vee\ (n.derecho).nivel = n.nivel)) \wedge$
\comentario{5} $((\forall n:\ nodo(\kappa, \sigma))(\ n \in\ arbol(e) \wedge n.derecho \neq NULO \wedge_L (n.derecho).derecho \neq NULO \Rightarrow_L ((n.derecho).derecho).nivel < n.nivel) \wedge$
\comentario{6} $((\forall n:\ nodo(\kappa, \sigma))(\ n \in\ arbol(e) \wedge n.nivel > 1 \Rightarrow\ (n.izquierdo \neq NULO \wedge n.derecho \neq NULO)$


$\ $ \newline $\ $
\par Abs: $estr d \rightarrow\ dicc(\kappa, \sigma)$ \{Rep($d$) \}
$\ $\newline \indent Abs($d$) $\equiv$ c: dicc($\kappa, \sigma$) | $((\forall k:\ \kappa)(\ k \in claves(c) \Rightarrow (\exists n:\ estr(\kappa, \sigma))( n \in arbol(d)\wedge n.clave = k)) \wedge
\newline \indent ((\forall n:\ nodo(\kappa, \sigma))( n \in arbol(d) \Rightarrow\ n.clave \in claves(c)))) \wedge_L
\newline \indent (\forall n:\ nodo(\kappa, \sigma))( n \in arbol(d) \Rightarrow\ obtener(c,n.clave) =_{obs} n.significado)$

\newpage

\indent \textbf{Especificaci\'on de las operaciones auxiliares utilizadas para Rep y Abs}
\newline \indent arbol:$\ puntero(nodo(\kappa\, \sigma)) \rightarrow conj(puntero(nodo(\kappa,\sigma)))$

\tadAxioma{arbol(n)}{
\IF {$n.izq \neq null \wedge n.der \neq null$}
THEN {Ag($\&n,arbol(n.izq)\cup arbol(n.der)$)}
ELSE { \IF {$n.izq \neq null$}
		THEN {Ag($\&n,arbol(n.izq)$)}
		ELSE { \LIF\ $n.der \neq null$ \LTHEN\ Ag($\&n,arbol(n.der)$) \LELSE\ Ag($\&n,\emptyset$) \LFI}
		FI}
FI}

\end{Representacion}

\newpage

\begin{Algoritmos}

\begin{algorithm}
\caption{Implementaci\'on de Definido?}
\begin{algorithmic}[0]
\Function{iDefinido?}{in d: estr , in c: $\kappa$}{$\rightarrow$ res: bool}
	\comentario{B\'usqueda est\'andar en un ABB}
	\State nodoActual $\gets$ d \Comment{O(1)}
	\State res $\gets$ FALSE \Comment{O(1)}
	\While{$\neg$nil?(nodoActual) $\&\&$ $\neg$res} \Comment{El ciclo se ejecuta en el peor caso una cantidad de veces igual a la altura del arbol. Al ser auto-balanceado, su altura siempre sera O(log(n))}
		\If{$\pi_1$(raiz(nodoActual)) == c} \Comment{O(comparar($\kappa$))}
			\State res $\gets$ TRUE \Comment{O(1)}
		\Else
			\If{c < $\pi_1$(raiz(nodoActual))} \Comment{O(comparar($\kappa$))}
				\State nodoActual $\gets$ izq(nodoActual) \Comment{O(1)}
			\Else
				\State nodoActual $\gets$ der(nodoActual) \Comment{O(1)}
			\EndIf
		\EndIf
	\EndWhile
\EndFunction
\end{algorithmic}
\end{algorithm}

\begin{algorithm}
\caption{Implementaci\'on de Significado}
\begin{algorithmic}[0]
\Function{iSignificado}{in d: estr , in c: $\kappa$}{$\rightarrow$ res: $\sigma$}
	\comentario{B\'usqueda est\'andar en un ABB}
	\State nodoActual $\gets$ d \Comment{O(1)}
	\While{$\neg$nil?(nodoActual) $\&\&$ $\neg$res} \Comment{El ciclo se ejecuta en el peor caso O(log(n)) veces.}
		\If{$\pi_1$(raiz(nodoActual)) == c} \Comment{O(comparar($\kappa$))}
			\State res $\gets$ $\pi_3$(raiz(nodoActual)) \Comment{O(copiar($\sigma$)). Esta operacion solo se ejecuta una vez (implica $\neg$guarda del ciclo que la contiene).}
		\Else
			\If{c < $\pi_1$(raiz(nodoActual))} \Comment{O(comparar($\kappa$))}
				\State nodoActual $\gets$ izq(nodoActual) \Comment{O(1)}
			\Else
				\State nodoActual $\gets$ der(nodoActual) \Comment{O(1)}
			\EndIf
		\EndIf
	\EndWhile
\EndFunction
\end{algorithmic}
\end{algorithm}

\begin{algorithm}
\caption{Implementaci\'on de Vacio}
\begin{algorithmic}[0]
\Function{iVacio}{}{$\rightarrow$ res: estr}
	\State res $\gets$ nil() \Comment{O(1)}
\EndFunction
\end{algorithmic}
\end{algorithm}

\begin{algorithm}
\caption{Implementaci\'on de Definir}
\begin{algorithmic}[0]
\Function{iDefinir}{inout d: estr , in c: $\kappa$, in s: $\sigma$}{}
	\comentario{Si ya se llego a una hoja, se inserta el nuevo elemento}
	\If{nil?(d)} \Comment{O(1)}
		\State $res\gets \ <c, NULO, NULO, s, 1>$ \Comment{O(max(copiar($\kappa$), copiar($\sigma$))}
	\comentario{Se busca la posicion correspondiente al nuevo nodo (antes de rebalancear el arbol).}
	\ElsIf{c < $\pi_1$(raiz(d))} \Comment{O(comparar($\kappa$)}
		\State setearIzq(d, iDefinir(izquierdo(d),c,s)) \Comment{En el peor caso se llama recursivamente a la funcion una cantidad de veces igual a la altura
		del arbol, que es O(log(n)).}
	\ElsIf{c > $\pi_1$(raiz(d))} \Comment{O(comparar($\kappa$))}
		\State setearDer(d, iDefinir(d.derecho,c,s)) \Comment{En el peor caso se llama recursivamente a la funcion una cantidad de veces igual a la altura
		del arbol, que es O(log(n)).}
	\EndIf
	\comentario{Se tuerce y divide el arbol en cada nivel, rebalanceandolo.}
	\State d $\gets$ Torsion(d) \Comment{O(1)}
	\State d $\gets$ Division(d) \Comment{O(1)}
\EndFunction
\end{algorithmic}
\end{algorithm}

\begin{algorithm}
\caption{Implementaci\'on de Torsion}
\begin{algorithmic}[0]
\Function{iTorsion}{in d: estr}{$\rightarrow$ res: estr}
	\comentario{Si el nodo tiene un hijo izquierdo del mismo nivel se debe realizar una rotacion para restaurar el invariante.}
	\If{nil?(d) $\|$ nil?(izq(d))} \Comment{O(1)}
		\State res $\gets$ d \Comment{O(1)}
	\Else
	\comentario{El hijo izquierdo de mismo nivel pasa a ser el padre del nodo derecho. El hijo derecho del nodo izquierdo pasa a ser el hijo izquierdo del nodo derecho.}
		\If{$\pi_2$(raiz(izq(d))) == $\pi_2$(d)} \Comment{O(1)}
			\State nodoAux $\gets$ izq(d) \Comment{O(1)}
			\State setearIzq(d, der(nodoAux)) \Comment{O(1)}
			\State setearDer(nodoAux, d) \Comment{O(1)}
			\State res $\gets$ nodoAux \Comment{O(1)}
		\Else
			\State res $\gets$ d \Comment{O(1)}
		\EndIf
	\EndIf
\EndFunction
\end{algorithmic}
\end{algorithm}

\begin{algorithm}
\caption{Implementaci\'on de Division}
\begin{algorithmic}[0]
\Function{iDivision}{in d: estr}{$\rightarrow$ res: estr}
	\comentario{Si hay dos hijos derechos del mismo nivel que el padre se debe realizar una rotacion para restaurar el invariante.}
	\If{nil?(d) $\|$ nil?(der(d)) $\|$ nil?(der(der(d)))} \Comment{O(1)}
		\State res $\gets$ d \Comment{O(1)}
	\Else
	\comentario{El primero hijo derecho pasa a ser el padre, con un nivel mas. Su hijo izquierdo pasa a ser el hijo derecho de su padre original.}
		\If{(d.derecho.derecho).nivel == d.nivel} \Comment{O(1)}
			\State nodoAux $\gets$ der(d) \Comment{O(1)}
			\State setearDer(d, izq(nodoAux)) \Comment{O(1)}
			\State setearIzq(nodoAux, d) \Comment{O(1)}
			\State $\pi_2$(raiz(nodoAux)) $\gets$ $\pi_2$(raiz(nodoAux))++ \Comment{O(1)}
			\State res $\gets$ nodoAux \Comment{O(1)}
		\Else
			\State res $\gets$ d \Comment{O(1)}
		\EndIf
	\EndIf
\EndFunction
\end{algorithmic}
\end{algorithm}

\FloatBarrier
\end{Algoritmos}

\begin{algorithm}
\caption{Implementaci\'on de Borrar}
\begin{algorithmic}[0]
\Function{iBorrar}{inout d: estr, in c: $\kappa$}{}
	\If{nil?(d)} \Comment{O(1)}
		\State $endFunction$
	\comentario{Se busca recursivamente la posicion del elemento a borrar mediante una busqueda estandar en ABB.}
	\ElsIf{c > $\pi_1$(raiz(d))} \Comment{O(comparar($\kappa$))}
		\State setearDer(d, iBorrar(der(d), c)) \Comment{En el peor caso se llama recursivamente a la funcion una cantidad de veces igual a la altura
		del arbol, que es O(log(n)).}
	\ElsIf{c < $\pi_1$(raiz(d))} \Comment{O(comparar($\kappa$))}
		\State setearIzq(d, iBorrar(izq(d), c)) \Comment{En el peor caso se llama recursivamente a la funcion una cantidad de veces igual a la altura
		del arbol, que es O(log(n)).}
	\comentario{Si el elemento a borrar es una hoja, simplemente se lo borra.}
	\ElsIf{nil?(izq(d)) $\wedge$ nil?(der(d))} \Comment{O(1)}
		\State $borrar(d)$ \Comment{O(max(borrar($\kappa$), borrar($\sigma$)))}
		\State $d\gets\ NULO$ \Comment{O(1)}
	\comentario{Si el elemento a borrar no es una hoja, se reduce al caso hoja.}
	\ElsIf{nil?(izq(d))} \Comment{O(1)}
	\comentario{Se busca el sucesor del elemento (bajando una vez por la rama izquierda y luego por la derecha hasta encontrar una hoja).}
		\State aux $\gets$ der(d) \Comment{O(1)}
		\While{$\neg$nil?(izq(aux))} \Comment{En el peor caso el ciclo se ejecuta O(log(n)) veces.}
			\State aux $\gets$ izq(aux) \Comment{O(1)}
		\EndWhile
		\comentario{Se hace un swap y se elimina el elemento.}
		\State setearDer(d, iBorrar($\pi_1$(raiz(aux)), der(d)))
		\State $\pi_1$(raiz(d)) $\gets$ $\pi_1$(raiz(aux)) \Comment{O(copiar($\kappa$))}
		\State $\pi_3$(raiz(d)) $\gets$ $\pi_3$(raiz(aux)) \Comment{O(copiar($\sigma$))}
	\Else
	\comentario{Se busca el predecesor del elemento (bajando una vez por la rama derecha y luego por la izquierda hasta encontrar una hoja).}
		\State aux $\gets$ izq(d) \Comment{O(1)}
		\While{$\neg$nil?(der(aux))} \Comment{En el peor caso el ciclo se ejecuta O(log(n)) veces.}
			\State aux $\gets$ der(aux) \Comment{O(1)}
		\EndWhile
		\comentario{Se hace un swap y se elimina el elemento.}
		\State setearIzq(d, iBorrar($\pi_1$(raiz(aux)), izq(d)))
		\State $\pi_1$(raiz(d)) $\gets$ $\pi_1$(raiz(aux)) \Comment{O(copiar($\kappa$))}
		\State $\pi_3$(raiz(d)) $\gets$ $\pi_3$(raiz(aux)) \Comment{O(copiar($\sigma$))}
	\EndIf
	\comentario{Se nivela, divide y tuerce para restaurar el invariante.}
	\State d $\gets$ Nivelar(T) \Comment{O(1)}
	\State d $\gets$ Torsion(T) \Comment{O(1)}
	\If{$\neg$nil?(der(d))} \Comment{O(1)}
		\State setearDer(der(d), Torsion(der(der(d))) \Comment{O(1)}
	\EndIf
	\State d $\gets$ Division(T) \Comment{O(1)}
	\State setearDer(d, Division(der(d)) \Comment{O(1)}
	\State res $\gets$ d \Comment{O(1)}
\EndFunction
\Procedure{Nivelar}{inout d: estr}{}
	\State nivel\_correcto $\gets$ min($\pi_2$(raiz(izq(d))), $\pi_2$(raiz(der(d))))+1 \Comment{O(1)}
	\If{nivel\_correcto < $\pi_2$(raiz(d))} \Comment{O(1)}
		\State $\pi_2$(raiz(d)) $\gets$ nivel\_correcto \Comment{O(1)}
		\If{nivel\_correcto < $\pi_2$(raiz(der(d)))} \Comment{O(1)}
			\State $\pi_2$(raiz(der(d))) $\gets$ nivel\_correcto \Comment{O(1)}
		\EndIf
	\EndIf
\EndProcedure
\end{algorithmic}
\end{algorithm}




\end{document}
